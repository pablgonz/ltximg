% arara: latex
% arara: dvips
% arara: ps2pdf
% arara: clean: { extensions : [ aux, log, out, ps, dvi,listing] }
% ltximg: extrenv: {chao}
% ltximg: deltenv: {hola}
% ltximg: options: {force, clean = tkz, prefix=clip}
\begin{filecontents*}{images/content1.tex}
\begin{pspicture}(4,6)
\psset{unit=2cm}
\pslineByHand[linecolor=red](0,0)(0,2)(2,2)
(2,0)(0,0)(2,2)(1,3)(0,2)(2,0)
\end{pspicture}
\end{filecontents*}
\documentclass{article}
\usepackage{pstricks-add}

\begin{filecontents*}[overwrite]{images/content2.tex}
\begin{pspicture}(4,6)
\psset{unit=2cm}
\pslineByHand[linecolor=red](0,0)(0,2)(2,2)
(2,0)(0,0)(2,2)(1,3)(0,2)(2,0)
\end{pspicture}
\end{filecontents*}

\begin{document}

\begin{itemize}
  \item The individual entries are indicated with a black dot, a so-called bullet.
  \item The text in the entries may be of any length.
\end{itemize}

No tocar \verb|\begin{pspicture}| junto a \verb|\end{pspicture}| y no reemplazar
% \tikzpicture
%\graphicspath{

\begin{enumerate}
  \item The labels consists of sequential numbers.
  \item The numbers starts at 1 with every call to the enumerate environment.
\end{enumerate}

Probando el paquete \verb|pstricks-add| junto al entorno \verb|pspicture|

\begin{pspicture}(4,6)
\psset{unit=2cm}
\pslineByHand[linecolor=red](0,0)(0,2)(2,2)
(2,0)(0,0)(2,2)(1,3)(0,2)(2,0)
\end{pspicture}

verbatim
\begin{verbatim}
\begin{pspicture}(4,6)
\psset{unit=2cm}
\pslineByHand[linecolor=red](0,0)(0,2)(2,2)
(2,0)(0,0)(2,2)(1,3)(0,2)(2,0)
\end{pspicture}
\end{verbatim}

\pstVerb{ 1234321 srand }
\begin{pspicture}[showgrid](-2,-2)(2,2)
\psframe*[linecolor=blue,opacity=!Rand](2,2)
\psframe*[linecolor=red,opacity=!Rand](-1,-1)(1,1)
\psframe*[linecolor=green,opacity=!Rand](-2,-2)(0,0)
\end{pspicture}

No tocar \verb|\begin{preview}| junto a \verb|\end{preview}| y no reemplazar
van algunas linea
%<*remove>
Texto al final del archivo
%</remove>
siguen mas lineas
fin documento
\end{document}


after end document
\begin{pspicture}(4,6)
\psset{unit=2cm}
\pslineByHand[linecolor=red](0,0)(0,2)(2,2)
(2,0)(0,0)(2,2)(1,3)(0,2)(2,0)
\end{pspicture}
