% arara: latex : {shell: true, action: nonstopmode, options: "-recorder"}
% arara: dvips
% arara: ps2pdf : {options: "-dPDFSETTINGS=/prepress"}
%<*remove>
% ltximg : options : {arara,output = salida, imgdir = pics, prefix = env, eps,svg, subenv,force, clean = doc} 
% ltximg : deltenv : {filecontents} 
%</remove>
\documentclass{article}
\usepackage{pstricks-add}
\usepackage{filecontents}%
\begin{document}
No tocar \myverb|\begin{preview}| junto a \verb|\end{preview}| y no reemplazar

% inicia filecontents 1
Algo de texto
\begin{filecontents*}{content1.tex}
\begin{pspicture}(4,6)
\psset{unit=2cm}
\pslineByHand[linecolor=red](0,0)(0,2)(2,2)
(2,0)(0,0)(2,2)(1,3)(0,2)(2,0)
\end{pspicture}
\end{filecontents*}
sigue el entonro 

% verbatim
\begin{verbatim}
\begin{pspicture}(4,6)
\psset{unit=2cm}
\pslineByHand[linecolor=red](0,0)(0,2)(2,2)
(2,0)(0,0)(2,2)(1,3)(0,2)(2,0)
\end{pspicture}
my $ltxtags = join "|", map {quotemeta} sort { length($a)<=>length($b) } keys %reverse_tag;
\end{verbatim}

% filecontents 2
\begin{filecontents*}{content2.tex}
Some mo er den indictum 
\end{filecontents*}

% verbatim 
\begin{verbatim}
\begin{filecontents*}{content2.tex}
\end{verbatim}

Usando noltximg , no debería quitar psset
%<*noltximg> 
\psset{unit=2cm}
% some code
\pspicture*(4,6)
\pslineByHand[linecolor=red](0,0)(0,2)(2,2)
(2,0)(0,0)(2,2)(1,3)(0,2)(2,0)
\endpspicture
%</noltximg>


% Este psset si deberia ser eliminado Extraer sera fig 1
\psset{unit=2cm}
% some code
\begin{pspicture*}(4,6)
\pslineByHand[linecolor=red](0,0)(0,2)(2,2)
(2,0)(0,0)(2,2)(1,3)(0,2)(2,0)
\end{pspicture*}
 
% verbatim
\begin{verbatim}
\end{filecontents*}
\end{verbatim}
 
% filecontents 3
\begin{filecontents*}{joined2.tex}
\documentclass{article}%
\usepackage{pstricks-add}
\begin{document}
\begin{pspicture}(4,6)
\pslineByHand[linecolor=green](0,0)(0,2)(2,2)
\end{pspicture}
\end{document}
\end{filecontents*}
% sigue el documento

No extraer este entorno usando no preview
\begin{nopreview}
\begin{pspicture}(4,6)
\psset{unit=2cm}
\pslineByHand[linecolor=red](0,0)(0,2)(2,2)
(2,0)(0,0)(2,2)(1,3)(0,2)(2,0)
\end{pspicture}
\end{nopreview}

% Este sera fig 2
\begin{pspicture}(4,6)
\psset{unit=2cm}
\pslineByHand[linecolor=red](0,0)(0,2)(2,2)
(2,0)(0,0)(2,2)(1,3)(0,2)(2,0)
\end{pspicture}

Comentar \verb|\begin{pspicture}| y no reemplazar
ltximg tags
    \verb|%<*ltximg>|, \verb|%</ltximg>|, \verb|%<*noltximg>|, \verb'%</noltximg>',
settings pstricks, tikz
    \verb'\psset', \verb'\tikzset'
plain pstricks
    \verb'\pspicture', \verb'\pspicture*', \verb'\endpspicture',\verb'\psgraph', \verb'\endpsgraph',
latex pstricks
    \verb|\begin{pspicture}|, \verb|\end{pspicture}|, \verb|\begin{pspicture*}|, \verb|\end{pspicture*}|
% Este sera fig 3
texto fig3
\begin{pspicture}(4,6)
\psset{unit=2cm}
\pslineByHand[linecolor=red](0,0)(0,2)(2,2)
(2,0)(0,0)(2,2)(1,3)(0,2)(2,0)
\end{pspicture}
mas texto

% Este sera fig 4
texto en la linea \begin{pspicture}(4,6)
\psset{unit=2cm}
\pslineByHand[linecolor=red](0,0)(0,2)(2,2)
(2,0)(0,0)(2,2)(1,3)(0,2)(2,0)
\end{pspicture} fig 4 mas texto en la misma linea

van algunas linea
%<*remove>
Texto al final del archivo
%</remove>
siguen mas lineas
fin documento
\end{document}

IGNORAR
No tocar \verb|\begin{preview}| junto a \verb|\begin{preview}| y no reemplazar
    \verb'\psset', \verb'\tikzset', \verb'\pspicture', \verb'\endpspicture',
    \verb'\psgraph', \verb'\endpsgraph',

\begin{postscript}
\begin{pspicture}[showgrid](0,3)(10,10)
Se queda
\end{pspicture}
\end{postscript}
