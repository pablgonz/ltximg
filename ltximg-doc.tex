\documentclass[11pt]{article}
\usepackage[T1]{fontenc}%
\usepackage{libertine}%
\usepackage{booktabs}%
\usepackage{tabularx}%
\usepackage[usenames,dvipsnames,svgnames,table]{xcolor}
\usepackage[margin=0.85in,letterpaper]{geometry}
\usepackage[scaled=0.9]{inconsolata}
\usepackage{listings}

\definecolor{lightgrey}{rgb}{0.9,0.9,0.9}
\definecolor{darkgreen}{rgb}{0,0.6,0}
\definecolor{darkred}{rgb}{0.6,0,0}
\definecolor{myblue}{RGB}{20,105,176}
\definecolor{darkgreen}{rgb}{0,0.6,0}

\lstdefinelanguage{mytex}[LaTeX]{TeX}{
	columns=flexible,
	frame=single,
	framerule=0pt,%
    backgroundcolor=\color{gray!10},%
    xleftmargin=\fboxsep,%
    xrightmargin=\fboxsep,
    alsoletter={\\,*,\&},
    morekeywords={\\AtBeginDocument,
                \\RequirePackage,
				\\PreviewEnvironment,
                \&},  
    morekeywords=[2]{pspicture,
  				   verbatim,
                   table,
                   other,
                   tikzpicture,
                   postscript,
                   preview,
                   TRICKS,
					POSTRICKS,
					TIKZPICTURE,
					OTHER,
                   nopreview},
   morekeywords=[3]{\\begin,
   					\\pspicture,
   					\\TRICKS,
   					\\ENDTRICKS,
   					\\endpspicture,
   					\\end},  
  literate=*{\{}{{\textcolor{myblue}{\{}}}{1}
            {\}}{{\textcolor{myblue}{\}}}}{1}
            {[}{{\textcolor{myblue}{[}}}{1}
            {]}{{\textcolor{myblue}{]}}}{1},
}

\lstset{language=mytex}

\lstdefinestyle{mystyle1}{
  basicstyle=\small\ttfamily,
  keywordstyle=\bfseries\color{red},
  keywordstyle=[2]{\color{magenta}},
  keywordstyle=[3]{\color{blue}},
  commentstyle=\color{darkgreen}, 
  stringstyle=\color{orange},
  identifierstyle=\ttfamily,
  showstringspaces=true,
  breaklines=true,
  tabsize=4,
  columns=fullflexible,
  keepspaces=true,
}

\lstset{style=mystyle1}
\begin{document}
\title{{\huge\textsf{lxtimg}}\\export tikz|pstricks environments to image format\\  \small v. 1.0}
\author{Pablo Gonz\'{a}lez Luengo\\ \small \ttfamily pablgonz at yahoo dot com}
\date{\today}
\maketitle
\begin{abstract}
\noindent

\textsf{ltximg}\footnote{Thanks to Giuseppe Matarazzo for his kind help on testing the script.} is a \textsf{Perl} script that automates the process to export 
\textsf{tikzpicture} or \textsf{pspicture} environments to image formats (PDF, EPS, PPM, PNG).
\end{abstract}
\tableofcontents
\section{Required Software}

For the full operation of the script you need the following opensource programs
(available for windows and linux), external to \textsc{ctan} repositories.

\begin{itemize}
\item \textsf{Perl}.

\item \textsf{Ghostscript}.

\item \textsf{pdftops} (optional, for images in EPS format).

\item \textsf{pdftoppm} (optional, for images in PPM format).

\item \textsf{ImageMagick} (optional, for conversion images).
\end{itemize}
\newpage
\section{Run and options}
For \TeX Live or Mik\TeX\ users the syntax for \textsf{ltximg} script is simple:

\begin{lstlisting}
perl ltximg file.tex -options
\end{lstlisting}


\begin{table}[htp]
\caption{Options for ltximg}
\begin{tabularx}{\linewidth}{@{}>{\ttfamily} l>{\ttfamily} l >{\ttfamily}l X @{}}\\\toprule
\emph{name} & \emph{short} & \emph{default} & \emph{description}\\\midrule
--help      &  --h         &                & display help information and exit.\\
--version   &  --v         &      			  & display version information and exit.\\
--license   &  --li        &			      & display license information and exit.\\
--imageDir= &              &       images    & The dir for the created images.\\
--DPI=      &  --d         &  300            & Dots per inch for gs, pdftoppm and mogrify.\\
--IMO="..." &              &                 & Aditional options for mogrify (need double quotes).\\
--clear     &  --c       	&				  & Delete all temp files.\\
--xetex     & --xe			&				  & Create all image using xelatex (tikz and pstricks).\\
--luatex    & --lu			&				  & Create all image using lualatex (tikz).\\
--latex     & --la			&				  & Create all image using latex(pstricks).\\
--useppm    & --up			&				  & Create jpg and png using mogrify and ppm\\
--usemog    & --um			&				  & Create jpg and png (transparent) using mogrify and pdf\\
--margins=  & --m			& 0				  & Margins for pdfcrop.\\
--pdf       &    			&				  & Create .pdf files using gs.\\
--ppm       &    			&				  & Create .ppm files (need pdftoppm).\\
--eps       &    			&				  & Create .eps files (need pdftops).\\
--jpg       &    			&				  & Create .jpg files (deafult use gs).\\
--png       &    			&				  & Create .png files (deafult use gs).\\
--skip=     & --s			& skip			  & Name for skip environmet in input file.\\
--other=    & --o   		& other			  & Name for other export environmet.\\
--all	    & --a			&				  & Create pdf/jpg/png/eps image type.\\
\bottomrule
\end{tabularx}
\end{table}

\section{How it works}

The script works in two steps, but giving the same result, a new file whit only 
\emph{tikzpicture} or \emph{pspicture} or bot environments and a folder with the images from
these environments.

\subsection{Comment and ignore}

The first step \textsf{ltximg } script create a image dir calls \textsf{images} 
and create a copy for in file, processing is as follows, being assumed that our file is \texttt{test.tex}:

\begin{enumerate}
\item  Create a copy file called test-tmp.tex and put the problematic environments (verbatim, verbatim\*, lstlisting,
 LTXexample, Verbatim, comment, alltt, minted, tcblisting, xcomment and skip) inside the:

\begin{lstlisting}
\begin{nopreview}
...
\end{nopreview}
\end{lstlisting}

and: 
\begin{enumerate}
\item If the option is latex adds the following lines to the beginning of the test-fig.tex:

\begin{lstlisting}
\AtBeginDocument{
\RequirePackage[active,tightpage]{preview}
\PreviewEnvironment{pspicture}
\PreviewEnvironment{other}}
\end{lstlisting}

\item If options its xetex adds the following lines to the beginning of the test-fig.tex:

\begin{lstlisting}
\AtBeginDocument{
\RequirePackage[xetex,active,tightpage]{preview}
\PreviewEnvironment{tikzpicture}
\PreviewEnvironment{pspicture}
\PreviewEnvironment{other}}
\end{lstlisting}

\item And if no option is given, adds the following lines at the beginning of the test-fig.tex. This is the default for
lualatex and pdflatex.

\begin{lstlisting}
\AtBeginDocument{
\RequirePackage[pdftex,active,tightpage]{preview}
\PreviewEnvironment{tikzpicture}
\PreviewEnvironment{other}}
\end{lstlisting}

\end{enumerate}

\item Open test-tmp.tex and change the problematic words for verbatin in line or after \% symbol:

\begin{lstlisting}
\pspicture         => \TRICKS
\endpspicture      => \ENDTRICKS
\begin{pspicture   => \begin{TRICKS
\end{pspicture     => \end{TRICKS
\begin{postscript} => \begin{POSTRICKS}
\end{postscript}   => \end{POSTRICKS}
\begin{tikzpicture => \begin{TIKZPICTURE
\end{tikzpicture   => \end{TIKZPICTURE
\begin{other       => \begin{OTHER
\end{other         => \end{OTHER
\end{lstlisting}

and save file called test-fig.tex then runs (pdf/lua/xe)latex in \texttt{test-fig.tex} and \textsf{pdfcrop} in 
\texttt{test-fig.pdf}.
\end{enumerate}
\subsection{Split and convert}
If \textsf{ltximg} called with option \textsf{-pdf} or \textsf{-eps} or \textsf{-um} the file \texttt{test-fig.pdf}
is splitting in \texttt{test-fig-1.pdf, test-fig-2.pdf,\ldots} and puts them into \texttt{images} dir. The invoked behind this command is:

\begin{lstlisting}
gs -q -sDEVICE=pdfwrite -dPDFSETTINGS=/prepress -dNOPAUSE -dBATCH -sOutputFile=imageDir/test-fig-%d.pdf \
test-fig.pdf
\end{lstlisting}
and then processes the remaining options.

For example, if you use the option \textsf{-pdf -um} the command behind this is:

\begin{lstlisting}
mogrify -define png:format=png32 -define png:compression-filter=4 -quality 100 -transparent white \
-density 300 -format png *.pdf
\end{lstlisting}

And if you use the option \textsf{-pdf -up} the command behind this is:

\begin{lstlisting}
mogrify -quality 100 -define png:format=png32 -define png:compression-filter=4 -density 300 \
-format png *.ppm
\end{lstlisting}


\section{Creating other images format}
If you need to create other image formats we first need to generate the PPM format or PDF, then the procedure is simple
using the \textsf{ImageMagick} \texttt{convert} command, command usage is so:
\begin{lstlisting}
mogrify -format ext *.ppm
\end{lstlisting}
for TIFF use  
\begin{lstlisting}
mogrify -format tiff *.ppm
\end{lstlisting}

\end{document}
