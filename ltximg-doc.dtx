% arara: xelatex
% arara: xelatex
% arara: clean: { extensions: [ aux, log, out, ilg, ind, idx, toc, hd ] }
% \iffalse meta-comment
%<*internal>
\iffalse
%</internal>
%<*readme>
## LTXimg &ndash; LaTeX environments to image format

## Description

**ltximg** is a perl *script* that automates the process of extracting and converting
environments provided by **tikz**, **pstricks** and other packages from input file
to image formats in individual files using `ghostscript` and `poppler-utils`. Generates a file
with only extracted environments and other with environments converted to `\includegraphics`.

## Syntax

```bash
$ ltximg [<compiler>] [<options>] [--] <input file>.<tex|ltx>
```
## Usage

```bash
$ ltximg --latex  [<options>] <file.tex>
$ ltximg --arara  [<options>] <file.tex>
$ ltximg [<options>] <file.tex>
$ ltximg <file.tex>
```

If used without `[<compiler>]` and `[<options>]` the extracted environments are converted to `pdf` image format
and saved in the `/images` directory using `pdflatex` and `preview` package. Relative or absolute `paths` for files
and directories is not supported. If the last `[<options>]` take a *list separated by commas*, you need `--` at the end.

## Default environments extract

```bash
    pspicture    tikzpicture    pgfpicture    psgraph    postscript    PSTexample
```

## Options

```
                                                                    [default]
-h, --help            Display command line help and exit            [off]
-l, --license         Display GPL license and exit                  [off]
-v, --version         Display current version (1.7) and exit        [off]
-t, --tif             Create .tif files using ghostscript           [gs]
-b, --bmp             Create .bmp files using ghostscript           [gs]
-j, --jpg             Create .jpg files using ghostscript           [gs]
-p, --png             Create .png files using ghostscript           [gs]
-e, --eps             Create .eps files using poppler-utils         [pdftops]
-s, --svg             Create .svg files using poppler-utils         [pdftocairo]
-P, --ppm             Create .ppm files using poppler-utils         [pdftoppm]
-g, --gray            Gray scale for images using ghostscript       [off]
-f, --force           Capture "\psset" and "\tikzset" to extract    [off]
-n, --noprew          Create images files whitout "preview" package [off]
-d <integer>, --dpi <integer>
                      Dots per inch resolution for images           [150]
-m <integer>, --margin <integer>
                      Set margins for pdfcrop                       [0]
--imgdir <dirname>    Set name of directory to save images          [images]
--zip                 Compress files generated in .zip format       [off]
--tar                 Compress files generated in .tar.gz format    [off]
-o <filename>, --output <filename>
                      Create output file                            [off]
--verbose             Verbose printing                              [off]
--srcenv              Create files whit only code environment       [off]
--subenv              Create files whit preamble and code           [off]
--latex               Using latex>dvips>ps2pdf for compiler input
                      and pdflatex for compiler output              [off]
--dvips               Using latex>dvips>ps2pdf for compiler input
                      and latex>dvips>ps2pdf for compiler output    [off]
--arara               Use arara for compiler input and output       [off]
--xetex               Using xelatex for compiler input and output   [off]
--dvipdf              Using dvipdfmx for compiler input and output  [off]
--luatex              Using lualatex for compiler input and output  [off]
--prefix <string>     Set prefix append to each image file          [off]
--norun               Run script, but no create images files        [off]
--nopdf               Don't create a ".pdf" image files             [off]
--nocrop              Don't run pdfcrop                             [off]
--verbcmd <cmdname>   Set "\cmdname" verbatim command               [myverb]
--clean (doc|pst|tkz|all|off)
                      Removes specific text in output file          [doc]
--extrenv <env1,...>  Add new environments to extract               [empty]
--skipenv <env1,...>  Skip environments to extract                  [empty]
--verbenv <env1,...>  Add verbatim environments                     [empty]
--writenv <env1,...>  Add verbatim write environments               [empty]
--deltenv <env1,...>  Delete environments in output file            [empty]
```

## Example

```bash
$ ltximg --latex -e -p --srcenv --imgdir=mypics -o test-out test-in.ltx
```

```bash
$ ltximg --latex -ep --srcenv --imgdir mypics -o test-out  test-in.ltx
```

   Create a `/mypics` directory whit all extracted environments converted to
   image formats (`.pdf`, `.eps`, `.png`), individual files whit source code (`.tex`)
   for all extracted environments, a file `test-out.ltx` whit all environments converted to `\includegraphics`
   and file `test-in-fig-all.tex` with only the extracted environments using
   `latex>dvips>ps2pdf` and `preview` package for `<input file>` and `pdflatex`
   for `<output file>`.

## Documentation

   For full documentation use:

```bash
$ texdoc ltximg
```

   For recreation all documentation use:

```bash
$ arara ltximg-doc.dtx
```

## Licence

This program is free software; you can redistribute it and/or modify it under the terms of the GNU
General Public License as published by the Free Software Foundation; either version 3 of the License,
or (at your option) any later version.

This program is distributed in the hope that it will be useful, but WITHOUT ANY WARRANTY; without even
the implied warranty of MERCHANTABILITY or FITNESS FOR A PARTICULAR PURPOSE. See the GNU General Public
License for more details.

## Author

Written by Pablo González L <pablgonz@yahoo.com>, last update 2019-08-24.

## Copyright

Copyright 2013 - 2019 by Pablo González L
%</readme>
%<*internal>
\fi
\def\nameofplainTeX{plain}
\ifx\fmtname\nameofplainTeX\else
  \expandafter\begingroup
\fi
%</internal>
%<*internal>
\input docstrip.tex
\keepsilent
\askforoverwritefalse
\nopreamble\nopostamble
\generate{
  \file{README.md}{\from{\jobname.dtx}{readme}}
}
\ifx\fmtname\nameofplainTeX
  \expandafter\endbatchfile
\else
  \expandafter\endgroup
\fi
%</internal>
%<*documentation>
\documentclass{ltxdoc}
\usepackage[top=0.5in, bottom=0.5in, left=2in, right=1in,footskip=0.2in,%
            headsep=10pt]{geometry} % page dimension
\usepackage{unicode-math} %
\setmathfont[Scale = 0.95]{Latin Modern Math}
\setmainfont[
   Numbers           = OldStyle,
   Ligatures         = TeX,
   Scale             = 0.95,
   UprightFont       = *-Regular,
   ItalicFont        = *-Italic,
   BoldFont          = *-Bold,
   BoldItalicFont    = *-BoldItalic,
   SmallCapsFeatures = {Letters=SmallCaps},
   Extension =.otf]{LibertinusSerif}
\setsansfont[
   Numbers           = OldStyle,
   Ligatures         = TeX,
   Scale             = 0.95,
   UprightFont       = *-Regular,
   ItalicFont        = *-Italic,
   BoldFont          = *-Bold,
   SmallCapsFeatures = {Letters=SmallCaps},
   Extension = .otf]{LibertinusSans}
\setmonofont[
   Scale             = 0.80,
   Extension         = .otf,
   UprightFont       = *-Regular ,
   ItalicFont        = *-RegularIt,
   BoldFont          = *-Medium ,
   BoldItalicFont    = *-MediumIt
            ]{SourceCodePro} % source code font
\newfontfamily\lmmitalic{lmmono10-italic.otf}[
   Scale             = 0.95,%
   Extension         = .otf,%
   ItalicFont        = lmmono10-italic,%
   SmallCapsFont     = lmmonocaps10-oblique,%
   SlantedFont       = lmmonoslant10-regular,
   ]
\newfontfamily\fetamono{ffmw10.otf}[
   Scale             = 0.95,%
   RawFeature        ={+latn,+rand,+kern,+size},%
   ]
\newfontfamily\libertinusinitials{LibertinusSerifInitials-Regular.otf}
\usepackage{microtype,hologo} % LaTeX logo
\usepackage{enumitem,lastpage,microtype,titletoc} % custom
\usepackage[svgnames]{xcolor} %
\usepackage[sf,bf,compact,medium,pagestyles]{titlesec}
\usepackage{adjustbox,multicol,hyperref,xparse,listings,accsupp}
\usepackage{hyperxmp,imakeidx}%
\PageIndex
\EnableCrossrefs
\newcommand{\HP}[1]{\emph{\hyperpage{#1}}\normalsize}
\def\SortIndex#1#2{\index{#1\actualchar#2|HP}}
\indexsetup{level=\section,firstpagestyle=myheader}
%\makeindex[name=mydoc,options=-s gind.ist,columnsep=15pt,title={Index of Documentation}]
\makeindex[options=-s gind.ist,columnsep=15pt,title={Index of Documentation}]
% don't copy numbers in code example
\newcommand*{\noaccsupp}[1]{\BeginAccSupp{ActualText={}}#1\EndAccSupp{}}

% parindent
\setlength{\parindent}{0pt}

% Colors for options
\definecolor{optcolor}{rgb}{0.281,0.275,0.485}

% Identification
\def\myscript{ltximg}
\def\fileversion{1.7}
\def\filedate{2019-08-24}

% Logo whit libertuns and fetamono font
\newsavebox{\logobox}
\savebox{\logobox}{%
    \normalsize%
    {\libertinusinitials%
     \textcolor{red}{L}\hspace{-3.0pt}%
     \raisebox{-0.2em}{\small \textcolor{green}{T}}%
     \hspace{-2.9pt}\textcolor{blue}{X}}%
     \hspace{-1pt}\fetamono{\textcolor{gray}{img}}%
}%
\makeatletter
\newcommand{\LTXimg}{%
  \settoheight{\@tempdima}{L}%
  \resizebox{!}{\@tempdima}{\usebox{\logobox}}%
}
\makeatother

% email https://tex.stackexchange.com/a/663
\catcode`\_=11\relax%
\newcommand\email[1]{\_email #1\q_nil}%
\def\_email#1@#2\q_nil{%
  \href{mailto:#1@#2}{{\emailfont #1\emailampersat #2}}%
}%
\newcommand\emailfont{\sffamily}%
\newcommand\emailampersat{{\color{NavyBlue}\footnotesize@}}%
\catcode`\_=8\relax% %

% Config hyperref
\hypersetup{
  linkcolor          = blue!50,
  citecolor          = red!50,%
  urlcolor           = magenta,%
  colorlinks         = true,%
  pdftitle           = {.:: ltximg \fileversion{} (\filedate) --- LaTeX environments to image formats ::.},%
  pdfauthor          = {Pablo Gonz\'{a}lez Luengo},%
  pdfsubject         = {Documentation for version \fileversion},%
  pdfcopyright       = {\textcopyright 2019 by Pablo González Luengo},
  pdfcontacturl      = {https://github.com/pablgonz/ltximg},
  pdfkeywords        = {extract, conversion, images, tikz, pstricks},
  pdfstartview       = {FitH},%
  bookmarksopenlevel = 2,%
}

% Configuration titleps
\settitlemarks{section}
\renewpagestyle{plain}[\color{gray}\small\sffamily]{
\setfoot{}{}{\thepage/\pageref{LastPage}}}

\newpagestyle{myheader}[\color{gray}\small\sffamily]{
\renewcommand\makeheadrule{\color{gray}\rule[0.45\baselineskip]{\linewidth}{0.4pt}}
\setfoot{\scalebox{0.85}{\LTXimg}\space\textcopyright\space 2019 by Pablo González L}
        {}
        {\thepage/\pageref{LastPage}}
\sethead{\raisebox{0.75\baselineskip}{Documentation for version \fileversion\space[\filedate]}}
        {}
        {\raisebox{0.75\baselineskip}{\scshape\small\S.\thesection\space\sectiontitle}}
}
\setlength{\headheight}{21pt}%

% Table of contents
\titlecontents{section}[0mm]{}%
    {\bfseries\contentspush{\makebox[4mm][l]{\thecontentslabel\hfill}}}%
    {\hspace*{-4mm}}% numberless
    {\hspace{0.25em}\titlerule*[6pt]{.}\contentspage}%

\titlecontents{subsection}[4mm]{}%
    {\contentspush{\makebox[6mm][l]{\thecontentslabel\hfill}}}
    {\hspace*{-10mm}}% numberless
    {\hspace{0.25em}\titlerule*[6pt]{.}\contentspage}%

\titlecontents{subsubsection}[10mm]{}%
    {\contentspush{\makebox[8mm][l]{\thecontentslabel\hfill}}}
    {\hspace*{-18mm}}% numberless
    {\hspace{0.25em}\titlerule*[6pt]{.}\contentspage}%

\makeatletter
\renewcommand\tableofcontents{%
\begingroup%
\section*{\contentsname\quad{\color{gray}\leaders\hrule height 5pt depth -4.4pt\hfill}%
  \@mkboth{%
    \MakeUppercase\contentsname}{\MakeUppercase\contentsname}}%
\vspace*{-14pt}
\setlength{\columnsep}{10pt}%
 \begin{multicols}{2}%
    \@starttoc{toc}%
\end{multicols}%
\vspace*{-3pt}{\color{gray}\hrule height 0.6pt}%
\vspace*{5pt}
\endgroup
}
\makeatother

% Custom \meta[...]{...}, \marg[...]{...} and \oarg[...]{...} for color
\ExplSyntaxOn
%^^A user level commands
\RenewDocumentCommand{\meta}{O{}m}
  {
   \ltximg_meta_generic:Nnn \ltximg_meta:n { #1 } { #2 }
  }
\RenewDocumentCommand{\marg}{O{}m}
  {
   \ltximg_meta_generic:Nnn \ltximg_marg:n { #1 } { #2 }
  }
\RenewDocumentCommand{\oarg}{O{}m}
  {
   \ltximg_meta_generic:Nnn \ltximg_oarg:n { #1 } { #2 }
  }
%^^A variables and keys
\tl_new:N \l_ltximg_meta_font_tl

\keys_define:nn { ltximg/meta }
  {
   type .choice:,
   type / tt .code:n = \tl_set:Nn \l_ltximg_meta_font_tl { \ttfamily },
   type / rm .code:n = \tl_set:Nn \l_ltximg_meta_font_tl { \rmfamily },
   type .initial:n = tt,
   cf .tl_set:N = \l_ltximg_meta_color_tl,
   cf .initial:n = black,
   ac .tl_set:N = \l_ltximg_meta_anglecolor_tl,
   ac .initial:n = black,
   sbc .tl_set:N = \l_ltximg_meta_brackcolor_tl,
   sbc .initial:n = black,
   cbc .tl_set:N = \l_ltximg_meta_bracecolor_tl,
   cbc .initial:n = black,
  }
%^^A internal commands
\cs_new_protected:Npn \ltximg_meta_generic:Nnn #1 #2 #3
  {
   \group_begin:
    \keys_set:nn { ltximg/meta } { #2 }
    \color{ \l_ltximg_meta_color_tl }
    \l_ltximg_meta_font_tl
    #1 { #3 } % #1 is \ltximg_meta:n, \ltximg_marg:n or \ltximg_oarg:n
   \group_end:
  }
\cs_new_protected:Npn \ltximg_meta:n #1
  {
   \ltximg_meta_angle:n { \textlangle }
   \ltximg_meta_meta:n { #1 }
   \ltximg_meta_angle:n { \textrangle }
  }
\cs_new_protected:Npn \ltximg_marg:n #1
  {
   \ltximg_meta_brace:n { \textbraceleft }
   \ltximg_meta:n { #1 }
   \ltximg_meta_brace:n { \textbraceright }
  }
\cs_new_protected:Npn \ltximg_oarg:n #1
  {
   \ltximg_meta_brack:n { [ }
   \ltximg_meta:n { #1 }
   \ltximg_meta_brack:n { ] }
  }
\cs_new_protected:Npn \ltximg_meta_meta:n #1
  {
   \textnormal{\textit{#1}}
  }
\cs_new_protected:Npn \ltximg_meta_angle:n #1
  {
   \group_begin:
    \fontfamily{cmr}\selectfont
    \textcolor{\l_ltximg_meta_anglecolor_tl}{#1}
   \group_end:
  }
\cs_new_protected:Npn \ltximg_meta_brace:n #1
  {
   \group_begin:
    \color{\l_ltximg_meta_bracecolor_tl}
    #1
   \group_end:
  }
\cs_new_protected:Npn \ltximg_meta_brack:n #1
  {
   \textcolor{\l_ltximg_meta_brackcolor_tl}{#1}
  }

% \ltximg for body document
\DeclareDocumentCommand{\ltximg}{}
  {
   \normalsize\texttt{\bfseries\textcolor{NavyBlue}{ltximg}}
  }

% \prgname{sm} : #1 index compiler, #2 index programs:
\DeclareDocumentCommand{\prgname}{sm}
  {%
    \IfBooleanTF{#1}
     {
       \textcolor{ForestGreen}{\ttfamily\bfseries{#2}}
       \SortIndex{compiler}{Compiler>\small\texttt{#2}}
     }
     {
       \textcolor{ForestGreen}{\ttfamily\bfseries{#2}}
       \SortIndex{programs}{Programs>\small\texttt{#2}}
     }
  }%

% \prgopt{sm} : #1 compiler opt, #2 program opt:
\DeclareDocumentCommand{\prgopt}{sm}
  {%
    \IfBooleanTF{#1}
      {
        \textcolor{gray}{\ttfamily\bfseries{-{}#2}}
        \SortIndex{compiler  ~ options}{Compiler  ~ options>\small\texttt{-{}#2}}%
      }
      {
        \mbox{\texttt{-{}#2}}%
        \SortIndex{#2}{\small\texttt{-{}#2} (program ~ option)}%
      }
  }

% \scriptname*{m}
\DeclareDocumentCommand{\scriptname}{m}
  {
    \textcolor{ForestGreen}{\ttfamily\bfseries{#1}}
    \SortIndex{scripts}{Scripts>\small\texttt{#1}}%
  }

% \scriptopt{m}
\DeclareDocumentCommand{\scriptopt}{m}
  {
    \mbox{\texttt{#1}}
    \SortIndex{script ~ option}{Script ~ options>\small\texttt{#1}}%
  }

% \pkgname{m}
\DeclareDocumentCommand{\pkgname}{ m }
  {
    \textsf{\textcolor{SlateBlue}{#1}}
    \SortIndex{packages}{Packages>\small\texttt{#1}}
    \SortIndex{#1}{\texttt{#1} ~ (package)}
  }%

% \pkgopt{m}
\DeclareDocumentCommand{\pkgopt}{ m }
  {
    \textsf{\textcolor{Orange}{#1}}
    \SortIndex{package ~  options}{Package ~  options>\small\texttt{#1}}
    \SortIndex{#1}{\texttt{#1} (package  ~ option)}
  }

% \env{sm}, #1 not used now
\DeclareDocumentCommand{\env}{m}
  {
    \textcolor{optcolor}{\texttt{#1}}%
    \SortIndex{environment}{Environments>\small\texttt{#1}}%
  }

% \ics{sm}, #1 not used now
\DeclareDocumentCommand{\ics}{sm}
  {
    \textcolor{optcolor}{\ttfamily{\textbackslash#2}}%
    \SortIndex{#2}{\texttt{\small\textbackslash#2}}
  }

% file extention
\DeclareDocumentCommand{\fext}{m}
  {
    \mbox{\textcolor{optcolor}{\ttfamily\bfseries{.#1}}}%
    \SortIndex{files  ~ extention}{File  ~ extentions >\small\texttt{.#1}}%
  }

% image format/extention
\DeclareDocumentCommand{\iext}{m}
  {%
    \textcolor{optcolor}{\ttfamily\bfseries{#1}}%
    \SortIndex{Image format}{Image formats>\small\texttt{#1}}%
  }

% \sysydir{m}
\DeclareDocumentCommand{\sysdir}{m}
  {
    \mbox{\textcolor{NavyBlue}{\ttfamily{/#1}}}%
  }

% \sysfile{m} ...only for color in some examples
\DeclareDocumentCommand{\sysfile}{m}
  {
    \mbox{\textcolor{gray}{\ttfamily{#1}}}
  }

% \OSsystem{m} ...only for color in some examples
\DeclareDocumentCommand{\OSsystem}{m}
  {
    \mbox{\textcolor{NavyBlue}{\ttfamily\bfseries{#1}}}%
    \SortIndex{Operating ~ system}{Operating ~ system>\small\texttt{#1}}
  }

% \cmdopt[short]{long}
\DeclareDocumentCommand{\cmdopt}{om}
  {
    \IfNoValueTF{#1}
      {
        \textcolor{optcolor}{\ttfamily\bfseries{-\/-#2}}
      }
      {
        \textcolor{optcolor}{\ttfamily\bfseries{-{}#1}},
        \textcolor{optcolor}{\ttfamily\bfseries{-\/-#2}}
      }
    \SortIndex{options}{\textsf{\myscript}\ options ~ in ~ command ~ line>\small\texttt{-\/-#2}}%
  }

\ExplSyntaxOff
% \DescribeIF{m}, #1 image format
\newsavebox{\marginIF}
\NewDocumentCommand{\DescribeIF}{ m }
  {%
    \begin{lrbox}{\marginIF}%
      \begin{minipage}[t]{\marginparwidth}%
        \raggedleft
        \iext{#1}
      \end{minipage}%
    \end{lrbox}%
      \leavevmode%
      \marginpar{\usebox{\marginIF}}%
      \ignorespaces%
  }%

% \myenv{environ}
\DeclareDocumentCommand\myenv{m}
  {
    \moveright 0.0pt \hbox{%
    \begin{minipage}[t]{\marginparwidth}%
        \raggedleft\ttfamily%\small%
        {\textcolor{gray}{\textbackslash begin\{}}{\bfseries\textcolor{optcolor}{#1}}\textcolor{gray}{\}}\par%
        \meta[ac=gray,cf=gray]{env content}\par%
        {\textcolor{gray}{\textbackslash end\{}}{\bfseries\textcolor{optcolor}{#1}}\textcolor{gray}{\}}%
    \end{minipage}%
                    } % close hbox
    \SortIndex{Environment}{Environments suport by default>\small\texttt{#1}}%
  }

% \mytag{dtxtag}
\DeclareDocumentCommand\mytag{m}{%
\moveright 0.0pt \hbox{%
    \begin{minipage}[t]{\marginparwidth}%
        \raggedleft\ttfamily%\small%
        \textcolor{gray}{\%<*}{\bfseries\textcolor{optcolor}{#1}}\textcolor{gray}{>}\par%
        \meta[ac=gray,cf=gray]{content}\par%
        \textcolor{gray}{\%</}{\bfseries\textcolor{optcolor}{#1}}\textcolor{gray}{>}%
    \end{minipage}%
                    } % close hbox
    \SortIndex{docstrip}{Docstrip tag>\small\texttt{#1}}%
}%

% \DescribeTE{sm}, #1 tag, #2 env
\newsavebox{\marginenvtag}
\NewDocumentCommand\DescribeTE{sm}{%
\begin{lrbox}{\marginenvtag}%
   \begin{minipage}[t]{\marginparwidth}%
    \raggedleft
    \IfBooleanTF{#1}{\mytag{#2}}{\myenv{#2}}
    \end{minipage}%
\end{lrbox}%
    \leavevmode%
    \marginpar{\usebox{\marginenvtag}}%
    \ignorespaces%
}%

% DescribeOptFile*{options}{example}[!]
\newsavebox{\optinfile}
\NewDocumentCommand\DescribeOptFile{s m m O{\hphantom{!}}}{
\begin{lrbox}{\optinfile}%
   \begin{minipage}[t]{\marginparwidth}%
        \raggedleft\ttfamily\bfseries%
        \textcolor{optcolor}{\%{#4}\myscript\hspace*{1.5pt}}%
    \end{minipage}%
\end{lrbox}%
\leavevmode%
\marginpar{\usebox{\optinfile}}%
\lapbox[0pt]{-0.85\marginparsep}{\textcolor{red}{\texttt{:}}}%
\textcolor{optcolor}{\bfseries\texttt{{#2}}}%
\textcolor{red}{\hspace*{2.5pt}\texttt{:}}
\hspace*{-1pt}\marg[cbc=optcolor,ac=gray,cf=gray]{#3}
\vspace*{2pt}\par%
\IfNoValueTF{#1}
{}%%
{%
\SortIndex{options}{\textsf{\myscript}\ options in input file>\small\texttt{#2}}%
}%
}

% \DescribeCmd[...]{...}{...}{...}, need changue to \DescribeOptCmd
\newsavebox{\optcmdline}
\NewDocumentCommand\DescribeCmd{ommm}{
\begin{lrbox}{\optcmdline}%
   \begin{minipage}[t]{\marginparwidth}%
    \ttfamily\bfseries\raggedleft%
    \IfNoValueTF{#1}
    {\textcolor{optcolor}{-\/-#2}}
    {\textcolor{optcolor}{-{#1}}\textcolor{gray}{,} \textcolor{optcolor}{-\/-#2}}%
    \end{minipage}%
\SortIndex{options}{\textsf{\myscript}\ options in command line>\small\texttt{-\/-#2}}%
\end{lrbox}%
    \leavevmode%
    \marginpar{\usebox{\optcmdline}}%
    \ignorespaces
    \meta[ac=gray,cf=gray]{\textnormal{\sffamily{#3}}}
    \hfill\textcolor{gray}{\textsf{(default: {#4})}}%
    \vspace*{2pt}\par%
}

% Create a language for documentation
\lstdefinelanguage{ltximg-doc}{
    texcsstyle=*,%
    escapechar=`,%
    showstringspaces=false,%
    extendedchars=true, %
    stringstyle = {\color{red}},%
% comments
    morecomment=[l]{\%},%
    commentstyle=\lmmitalic\color{lightgray},%
% Important words 1
    keywordstyle=[1]{\color{NavyBlue}},%
    keywords=[1]{AtBeginDocument,begin,end,documentclass,BEGIN,END},%
% Other words 2
    keywordstyle=[2]{\color{blue!75}},%
    keywords=[2]{usepackage,graphicspath,RequirePackage,renewcommand,%
    PreviewBbAdjust,usetikzlibrary,tikzexternalize,psset,tikzset,PrependGraphicsExtensions,%
    DefineShortVerb,lstMakeShortInline,MakeSpecialShortVerb,UndefineShortVerb},%
% Other words 3
    keywordstyle=[3]{\color{optcolor!85}},%
    keywords=[3]{document,graphicx,preview,active,tightpage,article,grfext,description,filecontents,%
    external,tikz,clean,pst,tkz,eps,pdf,xetex,latex,luatex,dvips,png,srcenv,noprew,imgdir,prefix,output},%
% Reserved words 4(inputfile options)
    keywordstyle=[4]{\color{optcolor}},%
    keywords=[4]{ltximg,noltximg,remove,options,pspicture,endpspicture,%
    PSTexample,pgfpicture, endpgfpicture, tikzpicture, endtikzpicture, %
    psgraph, endpsgraph,nopreview,postscript, arara,extrenv,deltenv,skipenv},%
% Reserved in orange
    keywordstyle=[5]{\color{OrangeRed}},%
    keywords=[5]{images,includegraphics,env,file-out,pics,doc},%
% Reserved in orange
    keywordstyle=[6]{\color{red}},%
    keywords=[6]{verb,myverb},%
}[keywords,tex,comments,strings]% end languaje

% \begin{examplecode}[optlst]...\end{examplecode}
\lstnewenvironment{examplecode}[1][]{%
\lstset{
    language=ltximg-doc,%
    stringstyle = {\color{red}},%
    basicstyle=\ttfamily\small,%
    numbersep=1em,%
    numberstyle=\tiny\color{gray}\noaccsupp,%
    rulecolor=\color{gray!50},%
    framesep=\fboxsep,%
    framerule=\fboxrule,%
    xleftmargin=\dimexpr\fboxsep+\fboxrule\relax,%
    xrightmargin=\dimexpr\fboxsep+\fboxrule\relax,%
% literateee
    literate=*{\{}{{\bfseries\textcolor{gray}{\{}}}{1}
              {\}}{{\bfseries\textcolor{gray}{\}}}}{1}
              {[}{{\bfseries\textcolor{optcolor}{[}}}{1}
              {]}{{\bfseries\textcolor{optcolor}{]}}}{1}
              {*}{{\bfseries\textcolor{red}{*}}}{1}
              {:}{{\textcolor{red}{:}}}{1}
              {,}{{\textcolor{gray}{,}}}{1}
              {=}{{\textcolor{gray}{=}}}{1}
              {/}{{\textcolor{gray}{/}}}{1}
              {\%\ ltximg}{{\textcolor{gray}{\%}\space\bfseries\textcolor{optcolor}{ltximg}}}{8}
              {\%\ arara}{{\textcolor{gray}{\%}\space\bfseries\textcolor{optcolor}{arara}}}{7}
              {\{arara}{{\textcolor{gray}{\{arara}}}{6}
              {\%<*remove>}{{\bfseries\textcolor{gray}{\%<*remove>}}}{10}
              {\%</remove>}{{\bfseries\textcolor{gray}{\%</remove>}}}{10}
              {\%<*ltximg>}{{\bfseries\textcolor{gray}{\%<*ltximg>}}}{10}
              {\%</ltximg>}{{\bfseries\textcolor{gray}{\%</ltximg>}}}{10}
              {\%<*noltximg>}{{\bfseries\textcolor{gray}{\%<*noltximg>}}}{12}
              {\%</noltximg>}{{\bfseries\textcolor{gray}{\%</noltximg>}}}{12},%
          #1,%
    }% close lstset
}%
{}% close examplecode

% \begin{examplecmd}...\end{examplecmd}
\lstnewenvironment{examplecmd}{%
\lstset{
    language=ltximg-doc,%
    basicstyle=\ttfamily\small,%
    frame=single,%
    rulecolor=\color{gray!50},%
    framesep=\fboxsep,%
    framerule=\fboxrule,%
    xleftmargin=\dimexpr\fboxsep+\fboxrule\relax,%
    xrightmargin=\dimexpr\fboxsep+\fboxrule\relax,%
% Reserved words (cmd line options)
    classoffset=7,%
    keywordstyle=\bfseries\color{optcolor},%
    morekeywords={ltximg},%
% % Reserved words (cmd line options)
    classoffset=8,%
    keywordstyle={\bfseries\color{ForestGreen}},%
    morekeywords={gs,pdftoppm,pdftocairo,pdftops},%
% Only for command line options
    classoffset=5,%
    keywordstyle=\color{blue},%
    keywords={user,machine},%
    literate=*{[}{{\textcolor{darkgray}{[}}}{1}
              {]}{{\textcolor{darkgray}{]}}}{1}
              {@}{{\textcolor{blue}{@}}}{1}
              {\$}{{\textcolor{blue}{\$}}}{1}
              {:}{{\textcolor{blue}{:}}}{1}
              {§}{{\textcolor{red}{\$}}}{1}
              {~}{{\textcolor{blue}{\bfseries\textasciitilde}}}{1}%
    }% close lstset
}%
{}% close examplecmd

% \lstinline[style=inline]|...|
\lstdefinestyle{inline}
  {
   language=ltximg-doc,%
   basicstyle=\ttfamily\color{gray},%
   escapechar=`,%
   upquote=true,%
   literate=*{\%}{{\bfseries\textcolor{gray}{\%}}}{1}
  }

% set default style
\lstset{style=inline}


\begin{document}

\title{%
    {\fetamono latex environments }\\[3pt]%
    \scalebox{3.4}{\LTXimg}\\[2pt]%
    {\fetamono\addfontfeature{LetterSpace=12.0} to image format}\\%
    \Large
    v\fileversion{} --- \filedate\thanks{%
    This file describes a documentation for version \fileversion, last revised \filedate.}\\[25pt]%
    \author{%
    \large%
    \raisebox{-1pt}{\textcopyright}{}2013--2019 by Pablo González L%
    \thanks{E-mail: \texttt{\guillemotleft}\email{pablgonz@yahoo.com}\texttt{\guillemotright}}
    }%
\small
\textsc{ctan}: \url{http://www.ctan.org/pkg/ltximg}\\
\textsc{git}: \url{https://github.com/pablgonz/ltximg}
\vspace*{-2cm}
}%
\date{}
\maketitle

\begin{abstract}
\ltximg{} is a \prgname{perl} \emph{script} that automates the process of
extracting and converting environments provided by \pkgname{tikz}, %
\pkgname{pstricks} and other packages from \meta{input file} to image
formats in individual files using \prgname{ghostscript} and %
\prgname{poppler-utils}. Generates a file with only extracted environments
and another with environments converted to \ics{includegraphics}.
\end{abstract}

\tableofcontents
\setlength{\parskip}{3pt}

\section{Motivation}

The original idea was to extend the functionality of the \scriptname{pst2pdf}
script (only for \env{pspicture} and \env{postscript}) to work with %
\env{tikzpicture} and other environments.

The \pkgname{tikz} package allows to externalize the environments, but, the
idea was to be able to extend this to any type of environment covering three
central points:

\begin{enumerate}[font=\small , noitemsep,leftmargin=*]

\item Generate separate files for environments and converted into images.

\item Generate a file with only the extracted environments.

\item Generate a file replacing the environments by \ics{includegraphics}.
\end{enumerate}

From the side of \TeX{} there are some packages that cover several of these
points such as the \pkgname{preview}, \pkgname{xcomment}, \pkgname{external}
and \pkgname{cachepic} packages among others, but none covered all points.

In the network there are some solutions in \texttt{bash} that were able to
extract and convert environments, but in general they presented problems
when the document contained \emph{verbatim style} code or were only
available for \OSsystem{Linux}.

Analysed the situation the best thing was to create a new \emph{script} that
was able to cover the three points and was multi platform, the union of all
these ideas is born \ltximg. Finding the correct \emph{regular expressions}
and writing \emph{documentation} would be the great mission (which does not
end yet).

\thispagestyle{plain}
\newpage
\pagestyle{myheader}
\section{Required Software}\label{sec:software}

For the complete operation of \ltximg{} you need to have a modern %
\hologo{TeX} distribution such as \hologo{TeX}Live or \hologo{MiKTeX}, have
a version equal to or greater than \liningnums{5.28} of \prgname{perl}, a
version equal to or greater than \liningnums{9.24} of \prgname{ghostscript}
and have a version equal to or greater than \liningnums{0.52} of %
\prgname{poppler-utils}.

The distribution of \hologo{TeX}Live 2019 for \OSsystem{Windows} includes %
\ltximg{} and all requirements, \hologo{MiKTeX} users must install the
appropriate software for full operation.

The script has been tested on \OSsystem{Windows} (version 10) and %
\OSsystem{Linux} (fedora 30) in x64 architecture using \prgname{ghostscript} %
\liningnums{v9.26}, \prgname{poppler-utils} \liningnums{v0.52} to %
\liningnums{v0.73} and \prgname{perl} from \liningnums{v5.28} to %
\liningnums{v5.30}.

\section{How it works}

\label{sec:howtowork}

It is important to have a general idea of how the \emph{extraction and
conversion} process works and the requirements that must be fulfilled so
that everything works correctly, for this we must be clear about some
concepts related to how to work with the \meta{verbatim content}, the %
\meta{input file}, the \meta{output file} and the \meta{steps process}.

\subsection{The input file}

\label{sec:inputfile}

The \meta{input file} must comply with certain characteristics in order to
be processed, the content at the beginning and at the end of the \meta{input
file} is treated in a special way, before \lstinline|\documentclass| can only be
commented lines and after \lstinline|\end{document}| can go any type of content,
internally will split the \meta{input file} at this points.

If the \meta{input file} contains files using \ics{input} or \ics{include}
these will not be processed, from the side of the \emph{script} they only
represent lines within the file, if you want them to be processed it is
better to use the \scriptname{latexpand} first and then process the file.

Like \ics{input} or \ics{include}, blank lines, vertical spaces and tab
characters are treated literally, for the \emph{script} the \meta{input file}
is just a set of characters, as if it was a simple text file. It is
advisable to format the source code \meta{input file} using utilities such
as \prgname{chktex} and \scriptname{latexindent}, especially if you want to
extract the source code of the environments.

An example of the \meta{input file}:

\begin{examplecode}[numbers=left,frame=single]
% some commented lines at begin document
\documentclass{article}
\usepackage{tikz}
\begin{document}
Some text
\begin{tikzpicture}
Some code
\end{tikzpicture}
Always use \verb|\begin{tikzpicture}|
and \verb|\end{tikzpicture}| to open
and close environment
\begin{tikzpicture}
Some code
\end{tikzpicture}
Some text
\end{document}
% some lines after end document
\end{examplecode}

\subsection{Verbatim contents}\label{sec:verbatim}

One of the greatest capabilities of \ltximg{} script is to skip the complications
that \emph{verbatim style} content produces with the extraction of environments.
In order to skip the complications, the verbatim content is classified into
three types:

\begin{itemize}[nosep]
    \item Verbatim in line
    \item Verbatim standard
    \item Verbatim write
\end{itemize}

Each of these classifications works differently within the creation and
extraction process using different regular expressions for it.

\newpage

\subsubsection{Verbatim in line}

\label{sec:verbatim:inline}

The small pieces of code written in the same line using a verbatim command
are considered \meta{verbatim in line}, such as \lstinline+\verb|<code>|+.
Most verbatim commands provide by packages \pkgname{minted}, %
\pkgname{fancyvrb} and \pkgname{listings} have been tested and are fully
supported. They are automatically detected the verbatim command generates by
\ics{newmint} and \ics{newmintinline} and the following command list:

\begin{multicols}{3}
\begin{itemize}[font=\sffamily\small,partopsep=5pt,parsep=5pt,nosep,leftmargin=*]
\small
\item \ics{mint}
\item \ics{spverb}
\item \ics{qverb}
\item \ics{fverb}
\item \ics{verb}
\item \ics{Verb}
\item \ics{lstinline}
\item \ics{pyginline}
\item \ics{pygment}
\item \ics{Scontents}
\item \ics{tcboxverb}
\item \ics{mintinline}
\end{itemize}
\end{multicols}

Some packages define abbreviated versions for verbatim commands as %
\ics{DefineShortVerb}, \ics{lstMakeShortInline} and %
\ics{MakeSpecialShortVerb}, will be detected automatically if are declared
explicitly in \meta{input file}.

The following consideration should be kept in mind for some packages that
use abbreviations for verbatim commands, such as \pkgname{shortvrb} or %
\pkgname{doc} for example in which there is no explicit command in the
document by means of which the abbreviated form can be detected, for
automatic detection need to find \ics{DefineShortVerb} explicitly to process
it correctly. The solution is quite simple, just add in \meta{input file}:

\begin{examplecode}
\UndefineShortVerb{\|}
\DefineShortVerb{\|}
\end{examplecode}

depending on the package you are using. If your verbatim command is not
supported by default or can not detect, use the options described in \ref%
{sec:optline} and \ref{sec:optfile}.

\subsubsection{Verbatim standard}

\label{sec:verbatim:std}

These are the classic environments for writing code are considered %
\meta{verbatim standard}, such as \env{verbatim} and \env{lstlisting}
environments. The following list is considered as \meta{verbatim standard}
environments:

\begin{multicols}{4}
\begin{itemize}[font=\sffamily\small, noitemsep,leftmargin=*]
\ttfamily\small
\item Example
\item CenterExample
\item SideBySideExample
\item PCenterExample
\item PSideBySideExample
\item verbatim
\item Verbatim
\item BVerbatim
\item LVerbatim
\item SaveVerbatim
\item PSTcode
\item LTXexample
\item tcblisting
\item spverbatim
\item minted
\item listing
\item lstlisting
\item alltt
\item comment
\item chklisting
\item verbatimtab
\item listingcont
\item boxedverbatim
\item demo
\item sourcecode
\item xcomment
\item pygmented
\item pyglist
\item program
\item programl
\item programL
\item programs
\item programf
\item programsc
\item programt
\end{itemize}
\end{multicols}

They are automatically detected \meta{verbatim standard} environments generates by commands:

\begin{multicols}{2}
\begin{itemize}[font=\sffamily\small, noitemsep,leftmargin=*]
\small
\item \ics{DefineVerbatimEnvironment}
\item \ics{NewListingEnvironment}
\item \ics{DeclareTCBListing}
\item \ics{ProvideTCBListing}
\item \ics{lstnewenvironment}
\item \ics{newtabverbatim}
\item \ics{specialcomment}
\item \ics{includecomment}
\item \ics{newtcblisting}
\item \ics{NewTCBListing}
\item \ics{newverbatim}
\item \ics{NewProgram}
\item \ics{newminted}
\end{itemize}
\end{multicols}

If any of the \meta{verbatim standard} environments is not supported by
default or can not detected, you can use the options described in \ref%
{sec:optline} and \ref{sec:optfile}.

\subsubsection{Verbatim write}

\label{sec:verbatim:write}

Some environments have the ability to write external files or memory directly, these
environments are considered \meta{verbatim write}, such as \env{filecontents}
or \env{VerbatimOut} environments. The following list is considered as %
\meta{verbatim write} environments:

\begin{multicols}{3}
\begin{itemize}[font=\sffamily\small, noitemsep,leftmargin=*]
\ttfamily\small
\item scontents
\item filecontents
\item tcboutputlisting
\item tcbexternal
\item tcbwritetmp
\item extcolorbox
\item extikzpicture
\item VerbatimOut
\item verbatimwrite
\item filecontentsdef
\item filecontentshere
\item filecontentsdefmacro
\end{itemize}
\end{multicols}

They are automatically detected \meta{verbatim write} environments generates
by commands:

\begin{multicols}{2}
\begin{itemize}[font=\sffamily\small, noitemsep,leftmargin=*]
\small
\item \ics{renewtcbexternalizetcolorbox}
\item \ics{renewtcbexternalizeenvironment}
\item \ics{newtcbexternalizeenvironment}
\item \ics{newtcbexternalizetcolorbox}
\end{itemize}
\end{multicols}

If any of the \meta{verbatim write} environments is not supported by default
or can not detected, you can use the options described in \ref{sec:optline}
and \ref{sec:optfile}.

\subsection{Steps process}

\label{sec:steps:process}

For creation of the image formats, extraction of code and creation of an
output file, \ltximg{} need a various steps. Let's assume that the %
\meta{input file} is \sysfile{test.tex}, \meta{output file} is %
\sysfile{test-out}, the working directory are \sysdir{workdir}, the
directory for images are \sysdir{workdir/images} and the user's temporary
directory is \sysdir{tmp} and we want to generate images in \iext{pdf}
format together with the source codes of the environments.

\begin{description}[font=\sffamily\small,leftmargin=0em,style=nextline]
\item[Comment and ignore]
The first step is read and validated \oarg[type=rm,cf=gray,sbc=optcolor,ac=gray]{options} from the command
line and \sysfile{test.tex}, verifying that \sysfile{test.tex}, \sysfile{test-out} and the
directory \sysdir{images} are in \sysdir{workdir}, create the directory \sysdir{workdir/images} if it does
not exist and a temporary directory \sysdir{tmp/hG45uVklv9}. The entire file \sysfile{test.tex} is loaded
in memory and proceeds (in general terms) as follows:

\begin{quotation}
Search the words \lstinline|\begin{| and \lstinline|\end{| in verbatim standard, verbatim write, verbatim in line and
commented lines, if it finds them, converts to \lstinline|\BEGIN{| and \lstinline|\END{|, then places all code to
extract inside the \lstinline|\begin{preview}| \ldots \lstinline|\end{preview}|.
\end{quotation}

At this point all the code you want to extract is inside \lstinline|\begin{preview}| \ldots \lstinline|\end{preview}|
and the files \sysfile{test-fig-1.tex}, \sysfile{test-fig-2.tex}, \ldots{} are generated and saved in \sysdir{images}.

\item[Create random file]
In the second step, with the file already loaded in memory, creating a temporary file with a
random number (1981 for example) and proceed in two ways according to the \oarg[type=rm,cf=gray,sbc=optcolor,ac=gray]{options}
passed to the script:

\begin{enumerate}
\item If script is call \emph{whitout} \cmdopt[n]{noprew} options, adds the
following lines to the beginning of the \sysfile{test.tex} (in memory):

\begin{examplecode}
\AtBeginDocument{%
\RequirePackage[active,tightpage]{preview}
\renewcommand\PreviewBbAdjust{-60pt -60pt 60pt 60pt}}%
% rest of input file
\end{examplecode}
And save in a temporary file \sysfile{test-fig-1981.tex} in \sysdir{workdir}.

\item If script is call \emph{whit} \cmdopt[n]{noprew} options, all code to extract
its put inside the \env{preview} environment. The \lstinline|\begin{preview}|\ldots \lstinline|\end{preview}|
lines are only used as delimiters for extracting the content \emph{without} using the package \pkgname{preview}.

Creating a temporary file \sysfile{test-fig-1981.tex} in \sysdir{workdir}
whit the same preamble of \sysfile{test.tex} but the body only contains code that you want to extract.
\end{enumerate}

\item[Generate image formats]
In the third step the script run:
\begin{examplecmd}
[user@machine ~:]§`\small\meta[type=tt,cf=ForestGreen,ac=ForestGreen]{compiler}` -recorder -shell-escape `\small\sysfile{test-fig-1981.tex}`
\end{examplecmd}
generating the file \sysfile{test-fig-1981.pdf} whit all code extracted, move \sysfile{test-fig-1981.pdf}
to \sysdir{tmp/hG45uVklv9}, separate in individual files \sysfile{test-fig-1.pdf}, \sysfile{test-fig-2.pdf}, \ldots{}
and copy to \sysdir{workdir/images/}. The file \sysfile{test-fig-1981.tex} is moved to the \sysdir{workdir/images/}
and rename to \sysfile{test-fig-all.tex}.

Note the options passed to \meta[type=tt,cf=ForestGreen,ac=ForestGreen]{compiler} does not include \prgopt*{output-directory}
(it is not supported) and always use \prgopt*{recorder} \prgopt*{shell-escape} you must keep this in mind if you use \prgname{arara}.

\item[Create output file]
In the fourth step the script creates the output file \sysfile{test-out.tex} converting all extracted code to
\ics{includegraphics} and adding the following lines at end of preamble:

\begin{examplecode}[numbers=left]
\usepackage{graphicx}
\graphicspath{{images/}}
\usepackage{grfext}
\PrependGraphicsExtensions*{.pdf}
\end{examplecode}

If the packages \pkgname{graphicx} and \pkgname{grfext} are already loaded and the command \ics{graphicspath}
is found in the input file were detected automatically and only the changes will be added then proceed to run:
\begin{examplecmd}
[user@machine ~:]§`\small\meta[type=tt,cf=ForestGreen,ac=ForestGreen]{compiler}` -recorder -shell-escape `\small\sysfile{test-out.tex}`
\end{examplecmd}
generating the file \sysfile{test-out.pdf}.
\end{description}
Now the script read the files \sysfile{test-fig-1981.fls} and \sysfile{test-out.fls}, extract the information from the
temporary files generated in the process and then delete them together with the directory \sysdir{tmp/hG45uVklv9}.
An example for input and output file:

\begin{minipage}[c]{0.5\textwidth}
\begin{examplecode}[numbers=left]
\documentclass{article}
\usepackage{tikz}
\begin{document}
Some text
\begin{tikzpicture}
Some code
\end{tikzpicture}
Always use \verb|\begin{tikzpicture}|
and \verb|\end{tikzpicture}| to open
and close environment
\begin{tikzpicture}
some code
\end{tikzpicture}
Some text
\end{document}
\end{examplecode}
\begin{flushleft}
\sysfile{test.tex}
\end{flushleft}
\end{minipage}
\begin{minipage}[c]{0.5\textwidth}
\begin{examplecode}[numbers=left]
\documentclass{article}
\usepackage{tikz}
\usepackage{graphicx}
\graphicspath{{images/}}
\usepackage{grfext}
\PrependGraphicsExtensions*{.pdf}
\begin{document}
Some text
\includegraphics[scale=1]{test-fig-1}
Always use \verb|\begin{tikzpicture}|
and \verb|\end{tikzpicture}| to open
and close environment
\includegraphics[scale=1]{test-fig-2}
Some text
\end{document}
\end{examplecode}
\begin{flushleft}
\sysfile{test-out.tex}
\end{flushleft}
\end{minipage}

\section{Extract content}
\label{sec:extract}
The script provides two ways to extract content from \meta{input file}, using \meta[type=rm,cf=optcolor,ac=gray]{environments}
and \meta[type=rm,cf=optcolor,ac=gray]{docstrip tags}. Some environment (including a starred \texttt{\small\bfseries\textcolor{red}{*}} version)
are supported by default and if the environments are nested, the outermost will be extracted.

\subsection{Default environments}
\label{sec:extract:env}
\DescribeTE{preview}
Environment provide by \pkgname{preview} package. If \env{preview} environments
found in the input file will be extracted and converted these. Internally
converts all environments to extract in \env{preview} environments.
Is better comment this package in preamble unless the option \cmdopt[n]{noprew}{} is
used.

\vspace{\baselineskip}

\DescribeTE{pspicture}
Environment provide by \pkgname{pstricks} package. The plain
syntax \lstinline|\pspicture ... \endpspicture| its converted to
\lstinline|\begin{pspicture} ... \end{pspicture}|.
\vspace{\baselineskip}

\DescribeTE{psgraph}
Environment provide by \pkgname{pst-plot} package. The plain syntax \lstinline|\psgraph ... \endpsgraph|
its converted to \lstinline|\begin{psgraph} ... \end{psgraph}|.

\vspace{\baselineskip}

\DescribeTE{postscript}
Environment provide by \pkgname{pst-pdf} and \pkgname{auto-pst-pdf} packages.
Since the \pkgname{pst-pdf} and \pkgname{auto-pst-pdf} packages internally use
the \pkgname{preview} package, is better comment this in preamble.

\vspace{\baselineskip}

\DescribeTE{tikzpicture}
Environment provide by \pkgname{tikz} package. The plain syntax \lstinline|\tikzpicture ... \tikzpicture|
its converted to \lstinline|\begin{tikzpicture} ... \end{tikzpicture}|
but no a short \lstinline|\tikz...;|.
\vspace{\baselineskip}

\DescribeTE{pgfpicture}
Environment provide by \pkgname{pgf} package. Since the script uses a
\emph{recursive regular expression} to extract environments, no presents problems
if present \lstinline|pgfinterruptpicture|.
\vspace{\baselineskip}

\DescribeTE{PSTexample}
Environment provide by \pkgname{pst-exa} packages. The script automatically detects the
\lstinline|\begin{PSTexample}|  \lstinline|...\end{PSTexample}|
environments and processes them as separately compiled files. The user should have loaded the
package with the [\pkgopt{swpl}] or [\pkgopt{tcb}] option and run the script
using \cmdopt{latex}{} or \cmdopt{xetex}.

If you need to extract more environments you can use one of the options described in \ref{sec:optline} or \ref{sec:optfile}.

\subsection{Extract whit docstrip tags}

\label{sec:extract:tag}
\DescribeTE*{ltximg}
All content included between \lstinline|%<*ltximg> ... %</ltximg>| is extracted.
The tags can not be nested and should be at the beginning of the line and in separate lines.

\begin{examplecode}[frame=single]
% no space before open tag %<*
%<*ltximg>
code to extract
%</ltximg>
% no space before close tag %</
\end{examplecode}

\subsection{Prevent extraction and remove}\label{sec:noextract}
Sometimes you do not want to extract all the environments from \meta{input file} or you want to
remove environments or arbitrary content, for example auxiliary files to generate a graphic.
The script provides a convenient way to solve this situation.

\DescribeTE{nopreview}
Environment provide by \pkgname{preview} package. Internally the script
converts all no extract environments to \lstinline|\begin{nopreview} ... \end{nopreview}|.
Is better comment this package in preamble unless the option \cmdopt[n]{noprew}{} is used.
\vspace{\baselineskip}

\DescribeTE*{noltximg}
All content betwen \lstinline|%<*noltximg> ... %</noltximg>| are ignored and no
extract. The start and closing of the tag must be at the beginning of the line.

\begin{examplecode}[frame=single]
% no space before open tag %<*
%<*noltximg>
no extract this
%</noltximg>
% no space before close tag %</
\end{examplecode}

\DescribeTE*{remove}
All content betwen \lstinline|%<*remove> ... %</remove>| are deleted in the \meta{output file}. The start and closing
of the tag must be at the beginning of the line.

\begin{examplecode}[frame=single]
% no space before open tag %<*
%<*remove>
lines removed in output file
%</remove>
% no space before close tag %</
\end{examplecode}

If you want to remove specific environments automatically you can use one of
the options described in \ref{sec:optline} or \ref{sec:optfile}.

\section{Image Formats}\label{sec:image:format}
The \meta{image formats} generated by the \ltximg{} using \prgname{ghostscript}
and \textcolor{ForestGreen}{\ttfamily\bfseries poppler-utils}
are the following command lines:

\DescribeIF{pdf}
The image format generated using \prgname{ghostscript}. The line executed by the system is:

\begin{examplecmd}
[user@machine ~:]§ gs -q -dNOSAFER -sDEVICE=pdfwrite -dPDFSETTINGS=/prepress
\end{examplecmd}

\DescribeIF{eps}
The image format generated using \prgname{pdftoeps}. The line executed by the system is:

\begin{examplecmd}
[user@machine ~:]§ pdftops -q -eps
\end{examplecmd}

\DescribeIF{png}
The image format generated using \prgname{ghostscript}. The line executed by the system is:

\begin{examplecmd}
[user@machine ~:]§ gs -q -dNOSAFER -sDEVICE=pngalpha -r 150
\end{examplecmd}

\newpage

\DescribeIF{jpg}
The image format generated using \prgname{ghostscript}. The line executed by the system is:
\begin{examplecmd}
[user@machine ~:]§ gs -q -dNOSAFER -sDEVICE=jpeg -r 150 -dJPEGQ=100 \
                      -dGraphicsAlphaBits=4 -dTextAlphaBits=4
\end{examplecmd}

\DescribeIF{ppm}
The image format generated using \prgname{pdftoppm}. The line executed by the system is:

\begin{examplecmd}
[user@machine ~:]§ pdftoppm -q -r 150
\end{examplecmd}

\DescribeIF{tif}
The image format generated using \prgname{ghostscript}. The line executed by the system is:

\begin{examplecmd}
[user@machine ~:]§ gs -q -dNOSAFER -sDEVICE=tiff32nc -r 150
\end{examplecmd}

\DescribeIF{svg}
The image format generated using \prgname{pdftocairo}. The line executed by the system is:

\begin{examplecmd}
[user@machine ~:]§ pdftocairo -q -r 150
\end{examplecmd}

\DescribeIF{bmp}
The image format generated using \prgname{ghostscript}. The line executed by the system is:

\begin{examplecmd}
[user@machine ~:]§ gs -q -dNOSAFER -sDEVICE=bmp32b -r 150
\end{examplecmd}

\section{How to use}
\subsection{Syntax}
The syntax for \ltximg{} is simple:

\begin{examplecmd}
[user@machine ~:]§ ltximg `\small\meta[type=tt,cf=ForestGreen,ac=ForestGreen]{compiler} \oarg[type=tt,cf=gray,sbc=optcolor,ac=gray]{options} \textcolor{gray}{\texttt{[-\/-]}} \meta[type=tt,cf=OrangeRed,ac=OrangeRed]{file.ext}`
\end{examplecmd}

The extension \meta[type=tt,cf=OrangeRed,ac=OrangeRed]{ext} for \meta{input file} are \fext{tex} or \fext{ltx},
relative or absolute paths for files and directories is not supported. If used without \meta[type=tt,cf=ForestGreen,ac=ForestGreen]{compiler}
and \oarg[type=tt,cf=gray,sbc=optcolor,ac=gray]{options} the extracted environments are converted to \iext{pdf}
image format and saved in the \sysdir{images} directory using \prgname*{pdflatex} and \pkgname{preview} package.

\subsection{Options in command line}
\label{sec:optline}

\ltximg{} provides a \emph{command line interface} with short and long option names.
They may be given before the name of the file. Also, the order of specifying the
options is significant. Certain options accept a list separate by commas, this require a separated by
white space or equals sign \textcolor{red}{\texttt{=}} between option and list and if it's the last option
need \textcolor{red}{\texttt{-\/-}} at the end. Multiple short options can be bundling.

\DescribeCmd[h]{help}{bolean}{off}
Display a command line help text and exit.

\DescribeCmd[l]{license}{bolean}{off}
Display a license text and exit.

\DescribeCmd[v]{version}{bolean}{off}
Display the current version (\fileversion) and exit.

\DescribeCmd[d]{dpi}{int}{150}
Dots per inch for images files.

\DescribeCmd[t]{tif}{bolean}{off}
Create a .\iext{tif} images files using \prgname{ghostscript}.

\DescribeCmd[b]{bmp}{bolean}{off}
Create a .\iext{bmp} images files using \prgname{ghostscript}.

\DescribeCmd[j]{jpg}{bolean}{off}
Create a .\iext{jpg} images files using \prgname{ghostscript}.

\DescribeCmd[p]{png}{bolean}{off}
Create a .\iext{png} transparent image files using \prgname{ghostscript}.

\DescribeCmd[e]{eps}{bolean}{off}
Create a .\iext{eps} image files using \prgname{pdftops}.

\DescribeCmd[s]{svg}{bolean}{off}
Create a .\iext{svg} image files using \prgname{pdftocairo}.

\DescribeCmd[P]{ppm}{bolean}{off}
Create a .\iext{ppm} image files using \prgname{pdftoppm}.

\DescribeCmd[g]{gray}{bolean}{off}
Create a gray scale for all images using \prgname{ghostscript}. The line behind this options is:

\begin{examplecmd}
[user@machine ~:]§ gs -q -dNOSAFER -sDEVICE=pdfwrite -dPDFSETTINGS=/prepress        \
                      -sColorConversionStrategy=Gray -dProcessColorModel=/DeviceGray
\end{examplecmd}

\DescribeCmd[f]{force}{bolean}{off}
Try to capture \lstinline|\psset{...}| and \lstinline|\tikzset{...}| to extract.
When using the \cmdopt{force} option the script will try to capture \lstinline|\psset{...}| or
\lstinline|\tikzset{...}| and leave it inside the \env{preview} environment, any line that is between \lstinline|\psset{...}| and
\lstinline|\begin{pspicture}| or between \lstinline|\tikzset{...}| and \lstinline|\begin{tikzpicture}| will be captured.

\DescribeCmd[n]{noprew}{bolean}{off}
Create images files without \pkgname{preview} package. The \lstinline|\begin{preview}|\ldots \lstinline|\end{preview}|
lines are only used as delimiters for extracting the content \emph{without} using the package \pkgname{preview}.
Sometimes it is better to use it together with \cmdopt{force}.

\DescribeCmd[m]{margin}{numeric}{0}
Set margins in bp for \scriptname{pdfcrop}.

\DescribeCmd[o]{output}{output file name}{empty}
Create \meta{output file name} whit all extracted environments/contents converted to \ics{includegraphics}.
The \meta{output file name} must not contain extension.

\DescribeCmd{imgdir}{string}{images}
The name of directory for save images and source code.

\DescribeCmd{zip}{bolean}{off}
Compress only the files generated by the script during the process in \sysdir{images} in \fext{zip} format.
Does not include \meta{output file}.

\DescribeCmd{tar}{bolean}{off}
Compress only the files generated by the script during the process in \sysdir{images} in \fext{tar.gz} format.
Does not include \meta{output file}.

\DescribeCmd{verbose}{bolean}{off}
Show verbose information in screen and change \prgopt*{interaction} for compiler.

\DescribeCmd{srcenv}{bolean}{off}
Create separate files whit \emph{only code} for all extracted environments, is mutually exclusive whit \cmdopt{subenv}.

\DescribeCmd{subenv}{bolean}{off}
Create sub files whit \emph{preamble} and code for all extracted environments, is mutually exclusive whit \cmdopt{srcenv}.

\DescribeCmd{arara}{bolean}{off}
Use \prgname*{arara} for compiler files, need to pass \prgopt*{recorder} option to \meta{input file}:\par

\lstinline|% arara : <compiler> : { options: [-recorder] }|

\DescribeCmd{xetex}{bolean}{off}
Using \prgname*{xelatex} compiler \meta{input file} and \meta{output file}.

\DescribeCmd{latex}{bolean}{off}
Using \prgname*{latex}\texttt{\bfseries\guillemotright}\prgname*{dvips}\texttt{\bfseries\guillemotright}\scriptname{ps2pdf}
compiler in \meta{input file} and \prgname*{pdflatex} for \meta{output file}.

\DescribeCmd{dvips}{bolean}{off}
Using \prgname*{latex}\texttt{\bfseries\guillemotright}\prgname*{dvips}\texttt{\bfseries\guillemotright}\scriptname{ps2pdf}
for compiler \meta{input file} and \meta{output file}.

\newpage

\DescribeCmd{dvipdf}{bolean}{off}
Using \prgname*{latex}\texttt{\bfseries\guillemotright}\prgname*{dvipdfmx} for compiler \meta{input file} and \meta{output file}.

\DescribeCmd{luatex}{bolean}{off}
Using \prgname*{lualatex} for compiler \meta{input file} and \meta{output file}.

\DescribeCmd{prefix}{string}{fig}
Add prefix append to each files created.

\DescribeCmd{norun}{bolean}{off}
Run script, but no create images. This option is designed to debug the file and
when you only need to extract the code

\DescribeCmd{nopdf}{bolean}{off}
Don't create a .\iext{pdf} image files.

\DescribeCmd{nocrop}{bolean}{off}
Don't run \scriptname{pdfcrop} in image files.

\DescribeCmd{clean}{doc\textbar pst\textbar tkz\textbar all\textbar off}{doc}
Removes specific content in \meta{output file}. Valid values for this option are:

\begin{description}[font=\ttfamily, nosep, noitemsep, leftmargin=*]
\item[doc] All content after \lstinline+\end{document}+ is removed.
\item[pst] All \lstinline|\psset{...}| and \pkgname{pstricks} package  is removed.
\item[tkz] All \lstinline|\tikzset{...}| is removed.
\item[all] Activates doc, pst and tkz.
\item[off] Deactivate all.
\end{description}

\DescribeCmd{verbcmd}{command name}{myverb}
Set custom verbatim command \lstinline+\myverb|<code>|+.

\DescribeCmd{extrenv}{list separate by comma}{empty}
List of environments to extract, need \texttt{-\/-} at end.\par

\DescribeCmd{skipenv}{list separate by comma}{empty}

List of environments that should not be extracted and that the script supports
by default, need \texttt{-\/-} at end.\par

\DescribeCmd{verbenv}{list separate by comma}{empty}

List of \meta{verbatim standard} environment support, need \texttt{-\/-} at end.\par

\DescribeCmd{writenv}{list separate by comma}{empty}

List of \meta{verbatim write} environment support, need \texttt{-\/-} at end.\par

\DescribeCmd{deltenv}{list separate by comma}{empty}
List of environment deleted in \meta{output file}, need \texttt{-\/-} at end.

\subsection{Options in input file}\label{sec:optfile}

Many of the ideas in this section are inspired by the \prgname{arara} program (I adore it).
A very useful way to pass options to the script is to place them in commented
lines at the beginning of the file, very much in the style of \prgname{arara}.

\DescribeOptFile*{\meta[ac=gray,cf=gray]{argument}}{option one, option two, option three, \ldots}
\DescribeOptFile*{\meta[ac=gray,cf=gray]{argument}}{option one, option two, option three, \ldots}[!]

The vast majority of the options can be passed into the \meta{input file}. These
should be put at the beginning of the file in commented lines and everything must
be on the same line, the exclamation mark deactivates the option. Valid values for
\meta[ac=gray,cf=gray]{argument} are the following:

\DescribeOptFile{options}{option one = value, option two = value, option three = value, \ldots}
This line is to indicate to the script which options need to process.

\DescribeOptFile{extrenv}{environment one, environment two, environment three, \ldots}
This line is to indicate to the script which environments, not supported by
default, are extracted.

\DescribeOptFile{skipenv}{environment one, environment two, environment three, \ldots}
This line is to indicate to the script which environments, of the ones
supported by default, should not be extracted.

\DescribeOptFile{verbenv}{environment one, environment two, environment three, \ldots}
This line is to indicate to the script which environments, its considerate a \meta{verbatim standard}.

\DescribeOptFile{writenv}{environment one, environment two, environment three, \ldots}
This line is to indicate to the script which environments its consider \meta{verbatim write}.

\DescribeOptFile{deltenv}{environment one, environment two, environment three, \ldots}
This line is to indicate to the script which environments are deleted.

If you are going to create an \meta{output file} and you do not want these lines to remain, it is better to
place them inside the \lstinline|%<*remove> ... %</remove>|. Like this:

\begin{examplecode}[numbers=left]
%<*remove>
% ltximg : options : {png,srcenv,xetex}
% ltximg : extrenv : {description}
%</remove>
\end{examplecode}


\section{Examples}
\label{sec:examples}

\subsection{From command line}
\begin{examplecmd}
[user@machine ~:]§ ltximg --latex -s -o test-out  test-in.ltx
\end{examplecmd}
Create a \sysdir{images} directory whit all extracted environments converted to image
formats (\iext{pdf}, \iext{svg}) in individual files, an \meta{output file} \sysfile{test-out.ltx} whit all supported  environments
converted to \ics{includegraphics} and a single file \sysfile{test-in-fig-all.ltx} with only the extracted environments
using \prgname*{latex}\texttt{\bfseries\guillemotright}\prgname*{dvips}\texttt{\bfseries\guillemotright}\scriptname{ps2pdf}
and \pkgname{preview} package for \meta{input file} and \prgname*{pdflatex} for \meta{output file}.

\subsection{From input file}
Adding the following lines to the beginning of the file \sysfile{file-in.tex}:

\begin{examplecode}[numbers=left]
%<*remove>
% ltximg : options : {output = file-out, noprew, imgdir = pics, prefix = env, clean = doc}
% ltximg : skipenv : {tikzpicture}
% ltximg : deltenv : {filecontents}
%</remove>
\end{examplecode}
and run:
\begin{examplecmd}
[user@machine~:]§ ltximg file-in.tex
\end{examplecmd}
Create a \sysdir{pics} directory whit all extracted environments, except \env{tikzpicture}, converted to image
formats (\iext{pdf}) in individual files, an \meta{output file} \sysfile{file-out.tex} whit all extracted  environments
converted to \ics{includegraphics} and environment \env{filecontents} removed, a single file \sysfile{test-in-env-all.ltx}
with only the extracted environments using \prgname*{pdflatex} and \pkgname{preview} package for \meta{input file} and \meta{output file}.

\newpage

\section{Change history}\label{sec:change:history}

Some of the notable changes in the history of the \ltximg{} along with the
versions, both development (devp) and public (ctan).

\setlist[itemize, 1]{label=\textendash, wide=0.5em,  nosep, noitemsep, leftmargin=10pt}

% length for change history
\newlength\descrwidth
\settowidth{\descrwidth}{\textsf{v1.4.5, (ctan), 2013-01-23}}
\begin{description}[font=\small\sffamily, wide=0pt, style=multiline, leftmargin=\descrwidth,  nosep, noitemsep]
\item [v\fileversion{} (ctan), \filedate]
    \begin{itemize}
        \item Add \texttt{scontents} environment support
        \item Add \texttt{filecontentsdefmacro} environment support
        \item Fix regex in source code
        \item Update documentation
    \end{itemize}
\item [v1.6 (ctan), 2019-07-13]
    \begin{itemize}
        \item Add \texttt{zip} and \texttt{tar} options
        \item Add new \texttt{Verb} from \texttt{fvextra}
        \item Fix and update source code and documentation
    \end{itemize}
\item [v1.5 (ctan), 2018-04-12]
    \begin{itemize}
        \item Use \texttt{GitHub} to control version
        \item Rewrite and optimize most part of source code and options
        \item Change \texttt{pdf2svg} for \texttt{pdftocairo}
        \item Complete support for \texttt{pst-exa} package
        \item Escape characters in regex according to \texttt{perl} v5.4x.x
    \end{itemize}
\item [v1.4 (devp), 2016-11-29]
    \begin{itemize}
        \item Remove and rewrite code for regex and system call
        \item Add \texttt{arara} compiler, clean and comment code
        \item Add \texttt{dvips} and \texttt{dvipdfm(x)} for creation images
        \item Add \texttt{bmp}, \texttt{tiff} image format
    \end{itemize}
\item [v1.3 (devp), 2016-08-14]
    \begin{itemize}
        \item Rewrite some part of code (norun, nocrop, clean)
        \item Suport \texttt{minted} and \texttt{tcolorbox} package
        \item Escape some characters in regex according to \texttt{perl} v5.2x.x
        \item All options read from command line and input file
        \item Use \texttt{/tmp} dir for work process
    \end{itemize}
\item [v1.2 (ctan), 2015-04-22]
    \begin{itemize}
        \item Remove unused modules
        \item Add more image format
        \item Fix regex
    \end{itemize}
\item [v1.1 (ctan), 2015-04-21]
    \begin{itemize}
        \item Change \texttt{mogrify} to \texttt{gs} for image formats
        \item Create output file
        \item Rewrite source code and fix regex
        \item Change format date to iso format
    \end{itemize}
\item [v1.0 (ctan), 2013-12-01]
    \begin{itemize}
        \item First public release
    \end{itemize}
\end{description}

\newpage

\addtocontents{toc}{\protect\setcounter{tocdepth}{2}}
\cleardoublepage
\phantomsection
%\printindex[mydoc]
%\PrintIndex
\printindex
\end{document}
%</documentation>
