% arara: lualatex
% arara: lualatex
% arara: clean: { extensions: [ aux, log, out, ilg, ind, idx, toc, hd, fls ] }
% \iffalse meta-comment
%<*internal>
\iffalse
%</internal>
%<*readme>
## LTXimg &ndash; LaTeX environments to image format

## Description

**ltximg** is a perl *script* that automates the process of extracting and converting
environments provided by **tikz**, **pstricks** and other packages from input file
to image formats and standalone files using `ghostscript` and `poppler-utils`. Generates a
file with only extracted environments and another with all extracted environments converted to `\includegraphics`.

## Syntax

```
$ ltximg [<compiler>] [<options>] [--] <input file>.<tex|ltx>
```

Relative or absolute `paths` for directories and files is not supported. Options that accept a _value_ require either a blank
space or `=` between the option and the _value_. Multiple short options can be bundling and if the last option takes a _comma
separated list_ you need `--` at the end.

## Usage

```
$ ltximg --latex  [<options>] <file.tex>
$ ltximg --arara  [<options>] <file.tex>
$ ltximg [<options>] <file.tex>
$ ltximg <file.tex>
```

If used without `[<compiler>]` and `[<options>]` the extracted environments are converted to `pdf` image format
and saved in the `/images` directory using `pdflatex` and `preview` package.

## Default environments extract

```
 preview  pspicture  tikzpicture  pgfpicture  psgraph  postscript  PSTexample
```

## Options

```
                                                                    [default]
-h, --help            Display command line help and exit            [off]
-v, --version         Display current version (1.8) and exit        [off]
-V, --verbose         Verbose printing information                  [off]
-l, --log             Write .log file with debug information        [off]
-t, --tif             Create .tif files using ghostscript           [gs]
-b, --bmp             Create .bmp files using ghostscript           [gs]
-j, --jpg             Create .jpg files using ghostscript           [gs]
-p, --png             Create .png files using ghostscript           [gs]
-e, --eps             Create .eps files using poppler-utils         [pdftops]
-s, --svg             Create .svg files using poppler-utils         [pdftocairo]
-P, --ppm             Create .ppm files using poppler-utils         [pdftoppm]
-g, --gray            Gray scale for images using ghostscript       [off]
-f, --force           Capture "\psset" and "\tikzset" to extract    [off]
-n, --noprew          Create images files whitout "preview" package [off]
-d <integer>, --dpi <integer>
                      Dots per inch resolution for images           [150]
-m <integer>, --margin <integer>
                      Set margins in bp for pdfcrop                 [0]
-o <filename>, --output <filename>
                      Create output file                            [off]
--imgdir <dirname>    Set name of directory to save generated files [images]
--prefix <string>     Set prefix append to each generated files     [fig]
--myverb <macroname>  Add "\macroname" to verbatim inline search    [myverb]
--clean (doc|pst|tkz|all|off)
                      Removes specific block text in output file    [doc]
--zip                 Compress files generated in .zip format       [off]
--tar                 Compress files generated in .tar.gz format    [off]
--srcenv              Create files whit only code environment       [off]
--subenv              Create files whit preamble and code           [off]
--latex               Using latex>dvips>ps2pdf for compiler input
                      and pdflatex for compiler output              [off]
--dvips               Using latex>dvips>ps2pdf for compiler input
                      and latex>dvips>ps2pdf for compiler output    [off]
--arara               Use arara for compiler input and output       [off]
--xetex               Using xelatex for compiler input and output   [off]
--dvipdf              Using dvipdfmx for compiler input and output  [off]
--luatex              Using lualatex for compiler input and output  [off]
--norun               Run script, but no create images files        [off]
--nopdf               Don't create a ".pdf" image files             [off]
--nocrop              Don't run pdfcrop                             [off]
--extrenv <env1,...>  Add new environments to extract               [empty]
--skipenv <env1,...>  Skip environments to extract                  [empty]
--verbenv <env1,...>  Add new verbatim environments                 [empty]
--writenv <env1,...>  Add new verbatim write environments           [empty]
--deltenv <env1,...>  Delete environments in output file            [empty]
```

## Example

```
$ ltximg --latex -e -p --srcenv --imgdir=mypics -o test-out test-in.ltx
$ ltximg --latex -ep --srcenv --imgdir mypics -o test-out.ltx  test-in.ltx
```

Create a `./mypics` directory (if it doesn’t exist) wwhit all extracted environments
converted to individual files (`.pdf`, `.eps`, `.png`, `.ltx`), a file `test-out.ltx`
whit all environments converted to `\includegraphics` and file `test-in-fig-all.ltx` with only the extracted environments using
`latex>dvips>ps2pdf` and `preview` package for `<input file>` and `pdflatex`
for `<output file>`.

## Documentation

For full documentation use:

```
$ texdoc ltximg
```

For recreation all documentation use:

```
$ arara ltximg-doc.dtx -H
```

## Licence

This program is free software; you can redistribute it and/or modify it under the terms of the GNU
General Public License as published by the Free Software Foundation; either version 3 of the License,
or (at your option) any later version.

This program is distributed in the hope that it will be useful, but WITHOUT ANY WARRANTY; without even
the implied warranty of MERCHANTABILITY or FITNESS FOR A PARTICULAR PURPOSE. See the GNU General Public
License for more details.

## Author

Written by Pablo González L <pablgonz@yahoo.com>, last update 2020-05-24.

## Copyright

Copyright 2013 - 2020 by Pablo González L
%</readme>
%<*internal>
\fi
\def\nameofplainTeX{plain}
\ifx\fmtname\nameofplainTeX\else
  \expandafter\begingroup
\fi
%</internal>
%<*internal>
\input docstrip.tex
\keepsilent
\askforoverwritefalse
\nopreamble\nopostamble
\generate{
  \file{README.md}{\from{\jobname.dtx}{readme}}
}
\ifx\fmtname\nameofplainTeX
  \expandafter\endbatchfile
\else
  \expandafter\endgroup
\fi
%</internal>
%<*documentation>
\documentclass{ltxdoc}
\usepackage[top=0.5in, bottom=0.3in, left=2in, right=0.7in,footskip=0.2in,headheight=1cm,headsep=0.27cm]{geometry}
\usepackage{unicode-math} %
\setmathfont[Scale = 0.95]{Latin Modern Math}
\setmainfont[
   Numbers           = OldStyle,
   Ligatures         = TeX,
   Scale             = 0.95,
   UprightFont       = *-Regular,
   ItalicFont        = *-Italic,
   BoldFont          = *-Bold,
   BoldItalicFont    = *-BoldItalic,
   SmallCapsFeatures = {Letters=SmallCaps},
   Extension =.otf]{LibertinusSerif}
\setsansfont[
   Numbers           = OldStyle,
   Ligatures         = TeX,
   Scale             = 0.95,
   UprightFont       = *-Regular,
   ItalicFont        = *-Italic,
   BoldFont          = *-Bold,
   SmallCapsFeatures = {Letters=SmallCaps},
   Extension = .otf]{LibertinusSans}
\setmonofont[
    Numbers        = OldStyle,
    Scale          = 0.80,
    UprightFont    = *-Regular,
    ItalicFont     = *-RegularIt,
    BoldFont       = *-Semibold,
    BoldItalicFont = *-SemiboldIt,
    RawFeature     = {+zero,+ss06},
    FontFace       = {m}{sc}{LinBiolinum_R.otf},
    Extension      = .otf]{SourceCodePro} % source code font
\newfontfamily\lmmitalic{lmmono10-italic.otf}[
   Scale             = 0.95,%
   Extension         = .otf,%
   ItalicFont        = lmmono10-italic,%
   SmallCapsFont     = lmmonocaps10-oblique,%
   SlantedFont       = lmmonoslant10-regular,
   ]
\newfontfamily\fetamontotf{ffmw10.otf}[
   Scale             = 0.95,%
   RawFeature        = {+latn,+rand,+kern,+size},%
   FontFace          = {bx}{n}{ffmw10.otf},
   ]
\newfontfamily\libertinusinitials{LibertinusSerifInitials-Regular.otf}
% The character of visible space is now taken from Latin Modern Mono
% to prevent fonts in T1. The original definition for xetex/luatex is
% \def\verbvisiblespace{\usefont{OT1}{cmtt}{m}{n}\asciispace}
\def\verbvisiblespace{{\fontfamily{lmtt}\selectfont\char"2423}}
\usepackage{fontawesome5}
\usepackage{hologo} % LaTeX logos
\usepackage{enumitem,lastpage,microtype,titletoc} % custom
\usepackage[svgnames]{xcolor} %
\usepackage[sf,bf,compact,medium]{titlesec} % section styles
\usepackage[topmarks,extramarks]{titleps} % pagestyles
\settitlemarks*{section,subsection}
\usepackage{adjustbox,multicol,hyperref,xparse,listings,accsupp}
\usepackage{csquotes}
\usepackage{hyperxmp,imakeidx}%
% Hyperref number in index
\newcommand{\HP}[1]{\emph{\hyperpage{#1}}\normalsize}
\def\SortIndex#1#2{\index{#1\actualchar#2|HP}}
\indexsetup{level=\section,firstpagestyle=myheader}
\makeindex[options=-s gind.ist,columnsep=15pt,title={Index of Documentation}]

% parindent
\setlength{\parindent}{0pt}

% Colors for options
\definecolor{optcolor}{rgb}{0.281,0.275,0.485}
\definecolor{pkgcolor}{rgb}{0,0.5,0.5}
\definecolor{araracolor}{rgb}{0, 0.72, 0.28}
\definecolor{linkcolor}{rgb}{0.04,0.38,0.04}
\definecolor{rulecolor}{rgb}{0.96,0.96,0.96}

% Only for arara... I like arara :)
\usepackage[scale=0.85]{comfortaa}
\newcommand{\araratext}[1]{{\small\normalfont\comfortaa\color{araracolor}\bfseries#1}}
\newcommand*\arara{\araratext{ar\kern-.03emar\kern-.03ema}}

% Identification
\def\myscript{ltximg}
\def\fileversion{1.8}
\def\filedate{2020-05-23}

% Logo with fetamont and libertuns font
\ExplSyntaxOn
\NewDocumentCommand{\pkglogo}{}
  {
    \group_begin:
       \libertinusinitials%
       \textcolor{red}{L}\hspace{-3.0pt}%
       \raisebox{-0.2em}{\small \textcolor{green}{T}}%
       \hspace{-2.9pt}\textcolor{blue}{X}%
       \hspace{-1pt}
       \fetamontotf{\textcolor{gray}{img}}
    \group_end:
  }
\ExplSyntaxOff

% Logo whit libertuns and fetamont font (box)
\newsavebox{\logobox}
\savebox{\logobox}{%
    \normalsize%
    {\libertinusinitials%
     \textcolor{red}{L}\hspace{-3.0pt}%
     \raisebox{-0.2em}{\small \textcolor{green}{T}}%
     \hspace{-2.9pt}\textcolor{blue}{X}}%
     \hspace{-1pt}\fetamontotf{\textcolor{gray}{img}}%
}%
\makeatletter
\newcommand{\LTXimg}{%
  \settoheight{\@tempdima}{L}%
  \resizebox{!}{\@tempdima}{\usebox{\logobox}}%
}
\makeatother

% email https://tex.stackexchange.com/a/663
\catcode`\_=11\relax%
\newcommand\email[1]{\_email #1\q_nil}%
\def\_email#1@#2\q_nil{%
  \href{mailto:#1@#2}{{\emailfont #1\emailampersat #2}}%
}%
\newcommand\emailfont{\sffamily}%
\newcommand\emailampersat{{\color{pkgcolor}\footnotesize@}}%
\catcode`\_=8\relax% %

% Config hyperref
\hypersetup{
  linkcolor          = linkcolor,%
  citecolor          = magenta,%
  urlcolor           = magenta,%
  colorlinks         = true,%
  linktoc            = all,
  pdftitle           = {.:: ltximg \fileversion{} (\filedate) --- LaTeX environments to image formats ::.},%
  pdfauthor          = {Pablo Gonz\'{a}lez Luengo},%
  pdfsubject         = {Documentation for version \fileversion},%
  pdfcopyright       = {\textcopyright 2013 - 2020 by Pablo González Luengo},
  pdfcontacturl      = {https://github.com/pablgonz/ltximg},
  pdfkeywords        = {extract, conversion, images, tikz, pstricks, standalone files},
  pdfstartview       = {FitH},%
  bookmarksopenlevel = 2,%
}

% Configuration titleps and titlesec
\renewpagestyle{plain}[\color{optcolor}\small\sffamily]{%
\setfoot{\rlap{\hskip\dimexpr-\oddsidemargin-1in\relax%
         \parbox{1.93\paperwidth}{\hfil\thepage\,/\,\pageref{LastPage}}}}%
        {}%
        {}%
}
\newpagestyle{myheader}[\color{optcolor}\small\sffamily]{%
\renewcommand\makeheadrule{%
\rlap{\hskip\dimexpr-\oddsidemargin-1in\relax
      \color{rulecolor}\rule[0.3cm]{\paperwidth}{0.7cm}}\hss
}
\setfoot{\rlap{\hskip\dimexpr-\oddsidemargin-1in\relax%
         \parbox{1.93\paperwidth}{\hfil\thepage\,/\,\pageref{LastPage}}}}%
        {\parbox{\textwidth}{\raggedright \textcolor{gray}{\raisebox{-1pt}{\textcopyright}{}2013 -- 2020 by Pablo González}}}%
        {}%
\sethead{\llap{\raisebox{0.55cm}{\parbox{\dimexpr\oddsidemargin+1in\relax}{\makebox[0pt][l]{\hspace{15pt}\pkglogo\space\fileversion}}}}}
        {\raisebox{0.55cm}{\parbox{\textwidth}{\hspace*{-\oddsidemargin}\centering\small\S.\firstextramarks{section}\thesection\space\sectiontitle}}}%
        {}%
}

% Table of contents
\titlecontents{section}[0mm]{}%
    {\bfseries\contentspush{\makebox[5mm][l]{\thecontentslabel\hfill}}}%
    {\hspace*{-5mm}}% numberless
    {\hspace{0.25em}\titlerule*[6pt]{.}\contentspage}%

\titlecontents{subsection}[5mm]{}%
    {\contentspush{\makebox[6mm][l]{\thecontentslabel\hfill}}}
    {\hspace*{-11mm}}% numberless
    {\hspace{0.25em}\titlerule*[6pt]{.}\contentspage}%

\titlecontents{subsubsection}[11mm]{}%
    {\contentspush{\makebox[8mm][l]{\thecontentslabel\hfill}}}
    {\hspace*{-19mm}}% numberless
    {\hspace{0.25em}\titlerule*[6pt]{.}\contentspage}%

\makeatletter
\renewcommand\tableofcontents{%
\begingroup%
\section*{\contentsname\quad{\color{optcolor}\leaders\hrule height 5pt depth -4.4pt\hfill}%
  \@mkboth{%
    \MakeUppercase\contentsname}{\MakeUppercase\contentsname}}%
\vspace*{-14pt}
\setlength{\columnsep}{10pt}%
 \begin{multicols}{2}%
    \@starttoc{toc}%
\end{multicols}%
\vspace*{-3pt}{\color{optcolor}\hrule height 0.6pt}%
\vspace*{5pt}
\endgroup
}
\makeatother

% Custom thanks
\makeatletter
\renewcommand*{\@fnsymbol}[1]{\ensuremath{\ifcase#1\or \textcolor{pkgcolor}{*}
\or \textcolor{pkgcolor}{\dagger} \or \ddagger\or \mathsection
\or \mathparagraph\or \|\or **\or \dagger\dagger
\or \ddagger\ddagger \else\@ctrerr\fi}}
\makeatother

% Custom \meta[...]{...}, \marg[...]{...} and \oarg[...]{...} for color
\ExplSyntaxOn
\RenewDocumentCommand{\meta}{O{}m}
  {
   \ltximg_meta_generic:Nnn \ltximg_meta:n { #1 } { #2 }
  }
\RenewDocumentCommand{\marg}{O{}m}
  {
   \ltximg_meta_generic:Nnn \ltximg_marg:n { #1 } { #2 }
  }
\RenewDocumentCommand{\oarg}{O{}m}
  {
   \ltximg_meta_generic:Nnn \ltximg_oarg:n { #1 } { #2 }
  }
%^^A Variables and keys
\tl_new:N \l_ltximg_meta_font_tl

\keys_define:nn { ltximg/meta }
  {
   type .choice:,
   type / tt .code:n = \tl_set:Nn \l_ltximg_meta_font_tl { \ttfamily },
   type / rm .code:n = \tl_set:Nn \l_ltximg_meta_font_tl { \rmfamily },
   type .initial:n = rm,
   cf .tl_set:N = \l_ltximg_meta_color_tl,
   cf .initial:n = optcolor,
   ac .tl_set:N = \l_ltximg_meta_anglecolor_tl,
   ac .initial:n = lightgray,
   sbc .tl_set:N = \l_ltximg_meta_brackcolor_tl,
   sbc .initial:n = gray,
   cbc .tl_set:N = \l_ltximg_meta_bracecolor_tl,
   cbc .initial:n = gray,
  }
%^^A Internal commands
\cs_new_protected:Npn \ltximg_meta_generic:Nnn #1 #2 #3
  {
   \group_begin:
    \keys_set:nn { ltximg/meta } { #2 }
    \color{ \l_ltximg_meta_color_tl }
    \l_ltximg_meta_font_tl
    #1 { #3 } % #1 is \ltximg_meta:n, \ltximg_marg:n or \ltximg_oarg:n
   \group_end:
  }
\cs_new_protected:Npn \ltximg_meta:n #1
  {
   \ltximg_meta_angle:n { \textlangle }
   \ltximg_meta_meta:n { #1 }
   \ltximg_meta_angle:n { \textrangle }
  }
\cs_new_protected:Npn \ltximg_marg:n #1
  {
   \ltximg_meta_brace:n { \textbraceleft }
   \ltximg_meta:n { #1 }
   \ltximg_meta_brace:n { \textbraceright }
  }
\cs_new_protected:Npn \ltximg_oarg:n #1
  {
   \ltximg_meta_brack:n { [ }
   \ltximg_meta:n { #1 }
   \ltximg_meta_brack:n { ] }
  }
\cs_new_protected:Npn \ltximg_meta_meta:n #1
  {
   \textnormal{\textit{#1}}
  }
\cs_new_protected:Npn \ltximg_meta_angle:n #1
  {
   \group_begin:
    \fontfamily{lmr}\selectfont
    \textcolor{\l_ltximg_meta_anglecolor_tl}{#1}
   \group_end:
  }
\cs_new_protected:Npn \ltximg_meta_brace:n #1
  {
   \group_begin:
    \color{\l_ltximg_meta_bracecolor_tl}
    #1
   \group_end:
  }
\cs_new_protected:Npn \ltximg_meta_brack:n #1
  {
   \textcolor{\l_ltximg_meta_brackcolor_tl}{#1}
  }

% \ltximg for body document
\NewDocumentCommand{\ltximg}{}
  {
   \normalsize\texttt{\textcolor{NavyBlue}{ltximg}}
  }

% \prgname{sm} : #1 index compiler, #2 index programs:
\NewDocumentCommand{\prgname}{sm}
  {
    \IfBooleanTF{#1}
     {
       \textcolor{ForestGreen}{\texttt{#2}}
       \SortIndex{compiler}{Compiler>\small\texttt{#2}}
     }
     {
       \textcolor{ForestGreen}{\texttt{#2}}
       \SortIndex{programs}{Programs>\small\texttt{#2}}
     }
  }

% \prgopt{sm} : #1 compiler opt, #2 program opt:
\NewDocumentCommand{\prgopt}{sm}
  {
    \IfBooleanTF{#1}
      {
        \mbox{\textcolor{gray}{\texttt{-{}#2}}}
        \SortIndex{compiler ~ options}{Compiler ~ options>\small\texttt{-{}#2}}
      }
      {
        \mbox{\textcolor{gray}{\texttt{-{}#2}}}
        \SortIndex{#2}{\small\texttt{-{}#2} (program ~ option)}%
      }
  }

% \scriptname*{m}
\NewDocumentCommand{\scriptname}{m}
  {
    \textcolor{ForestGreen}{\texttt{#1}}
    \SortIndex{scripts}{Scripts>\small\texttt{#1}}
  }

% \scriptopt{m}
\NewDocumentCommand{\scriptopt}{m}
  {
    \mbox{\texttt{#1}}
    \SortIndex{script ~ option}{Script ~ options>\small\texttt{#1}}
  }

% \myclass{m}
\NewDocumentCommand{\myclass}{ m }
  {
    \mbox{\textsf{\textcolor{SlateBlue}{#1}}}
    \SortIndex{document}{Document ~ class>\small\texttt{#1}}
    \SortIndex{#1}{\texttt{#1} ~ (class)}
  }

% \pkgname{m}
\NewDocumentCommand{\pkgname}{ m }
  {
    \textsf{\textcolor{SlateBlue}{#1}}
    \SortIndex{packages}{Packages>\small\texttt{#1}}
    \SortIndex{#1}{\texttt{#1} ~ (package)}
  }

% \pkgopt{m}
\NewDocumentCommand{\pkgopt}{ m }
  {
    \textsf{\textcolor{gray}{#1}}
    \SortIndex{package ~ options}{Package ~ options>\small\texttt{#1}}
    \SortIndex{#1}{\texttt{#1} ~ (package ~ option)}
  }

% \env{m}
\NewDocumentCommand{\env}{ m }
  {
    \textcolor{pkgcolor}{\texttt{#1}}
    \SortIndex{environment}{Environments>\small\texttt{#1}}
  }

% \ics{sm}
\NewDocumentCommand{\ics}{sm}
  {
    \IfBooleanTF{#1}
      {
        \tl_set:Nn \l_tmpa_tl { #2 }
        \regex_replace_once:nnN { (\*) } { \c{textcolor}\cB\{MediumOrchid\cE\}\cB\{\1\cE\} } \l_tmpa_tl
        \textcolor{blue!75}{\ttfamily\textbackslash{\tl_use:N \l_tmpa_tl}}
      }
      {
        \textcolor{blue!75}{\ttfamily{\textbackslash#2}}
      }
    \SortIndex{#2}{\texttt{\small\textbackslash#2}}
  }

% file extention
\NewDocumentCommand{\fext}{m}
  {
    \mbox{\textcolor{OrangeRed}{\texttt{.#1}}}
    \SortIndex{file ~ extention}{File ~ extentions >\small\texttt{.#1}}
  }

% image format/extention
\NewDocumentCommand{\iext}{m}
  {
    \textcolor{optcolor}{\texttt{#1}}
    \SortIndex{Image ~ format}{Image ~ formats>\small\texttt{#1}}
  }

% \sysydir{m}
\NewDocumentCommand{\sysdir}{m}
  {
    \mbox{\textcolor{OrangeRed}{\texttt{#1}}}
  }

% \sysfile{m} ...only for color in some examples
\NewDocumentCommand{\sysfile}{m}
  {
    \mbox{\textcolor{OrangeRed}{\texttt{#1}}}
  }

% \OSsystem{m} ...only for color in some examples
\NewDocumentCommand{\OSsystem}{m}
  {
    \mbox{\textcolor{OrangeRed}{\texttt{#1}}}
    \SortIndex{Operating ~ system}{Operating ~ system>\small\texttt{#1}}
  }

% \cmdopt[short]{long}
\NewDocumentCommand{\cmdopt}{om}
  {
    \IfNoValueTF{#1}
      {
        \textcolor{optcolor}{\texttt{-\/-#2}}
      }
      {
        \textcolor{optcolor}{\texttt{-{}#1}},
        \textcolor{optcolor}{\texttt{-\/-#2}}
      }
    \SortIndex{options}{\textsf{\myscript}\ options ~ in ~ command ~ line>\small\texttt{-\/-#2}}
  }

\ExplSyntaxOff
% \DescribeIF{m}, #1 image format
\newsavebox{\marginIF}
\NewDocumentCommand{\DescribeIF}{ m }
  {%
    \begin{lrbox}{\marginIF}%
      \begin{minipage}[t]{\marginparwidth}%
        \raggedleft\ttfamily\bfseries%
        \textcolor{optcolor}{#1}
      \end{minipage}%
    \end{lrbox}%
    \SortIndex{Image format}{Image formats>\small\texttt{#1}}
      \leavevmode%
      \marginpar{\usebox{\marginIF}}%
      \ignorespaces%
  }%

% \myenv{environ} for margin
\NewDocumentCommand{\myenv}{m}
  {
    \begin{minipage}[t]{\marginparwidth}%
        \raggedleft\ttfamily\small%
        {\textcolor{gray}{\textbackslash begin\{}}{\bfseries\textcolor{pkgcolor}{#1}}\textcolor{gray}{\}}\par%
        \meta[ac=lightgray,cf=gray]{env content}\par%
        {\textcolor{gray}{\textbackslash end\{}}{\bfseries\textcolor{pkgcolor}{#1}}\textcolor{gray}{\}}%
    \end{minipage}%
    \SortIndex{Environment}{Environments suport by default>\small\texttt{#1}}%
  }

% \mytag{dtxtag} for margin
\NewDocumentCommand{\mytag}{m}
  {%
   \begin{minipage}[t]{\marginparwidth}%
        \raggedleft\ttfamily\small%
        \textcolor{gray}{\%<*}{\bfseries\textcolor{pkgcolor}{#1}}\textcolor{gray}{>}\par%
        \meta[ac=lightgray,cf=gray]{content}\par%
        \textcolor{gray}{\%</}{\bfseries\textcolor{pkgcolor}{#1}}\textcolor{gray}{>}%
    \end{minipage}%
    \SortIndex{docstrip}{Docstrip tag>\small\texttt{#1}}%
  }%

% \DescribeTE{sm}, #1 tag, #2 env
\newsavebox{\marginenvtag}
\NewDocumentCommand{\DescribeTE}{sm}{%
\begin{lrbox}{\marginenvtag}%
   \begin{minipage}[t]{\marginparwidth}%
    \raggedleft
    \IfBooleanTF{#1}{\mytag{#2}}{\myenv{#2}}
    \end{minipage}%
\end{lrbox}%
    \leavevmode%
    \marginpar{\usebox{\marginenvtag}}%
    \ignorespaces%
}%

% DescribeOptFile*{options}{example}[!]
\newsavebox{\optinfile}
\NewDocumentCommand{\DescribeOptFile}{s m m O{\hphantom{!}}}{
\begin{lrbox}{\optinfile}%
   \begin{minipage}[t]{\marginparwidth}%
        \raggedleft\ttfamily\bfseries%
        \textcolor{gray}{\%}\textcolor{optcolor}{{#4}\myscript}\textcolor{gray}{\texttt{:}}%
    \end{minipage}%
\end{lrbox}%
\leavevmode%
\lapbox[0pt]{-\marginparwidth-0.5\marginparsep}{\usebox{\optinfile}}%
\textcolor{optcolor}{\bfseries\texttt{{#2}}}%
\textcolor{gray}{\texttt{:}}\space%
\marg[type=tt,cbc=gray,ac=lightgray,cf=optcolor!85]{#3}%
\vspace*{2pt}\par%
\IfBooleanF{#1}
  {%
    \SortIndex{options}{\textsf{\myscript}\ options in input file>\small\texttt{#2}}%
  }%
}

% \DescribeCmd[...]{...}{...}{...}
\newsavebox{\optcmdline}
\NewDocumentCommand{\DescribeCmd}{ o m m m }
  {
    \begin{lrbox}{\optcmdline}%
      \begin{minipage}[t]{\marginparwidth}%
      \ttfamily\bfseries\raggedleft%
      \IfNoValueTF{#1}
        {\textcolor{optcolor}{-\/-#2}}
        {\textcolor{optcolor}{-{#1}}\textcolor{gray}{,} \textcolor{optcolor}{-\/-#2}}%
      \end{minipage}%
      \SortIndex{options}{\textsf{\myscript}\ options in command line>\small\texttt{-\/-#2}}%
    \end{lrbox}%
    \leavevmode%
    \marginpar{\usebox{\optcmdline}}%
    \ignorespaces
    \meta[ac=lightgray,cf=gray]{#3}
    \hfill\textcolor{gray}{\textsf{(default: {#4})}}%
    \vspace*{2pt}\par%
}

% don't copy numbers in code example for listings
\newcommand*{\noaccsupp}[1]{\BeginAccSupp{ActualText={}}#1\EndAccSupp{}}

% Create a language for documentation
\lstdefinelanguage{ltximg-doc}{
    texcsstyle=*,%
    escapechar=`,%
    showstringspaces=false,%
    extendedchars=true, %
    stringstyle = {\color{red}},%
    alsoletter={\-,.},%
% comments
    morecomment=[l]{\%},%
    commentstyle=\lmmitalic\color{lightgray},%
% Important words 1
    keywordstyle=[1]{\color{NavyBlue}},%
    keywords=[1]{AtBeginDocument,documentclass,usepackage},%
% Important words 2 (macros)
    keywordstyle=[2]{\color{blue!75}},%
    keywords=[2]{graphicspath,RequirePackage,renewcommand,includegraphics,%
    PreviewBbAdjust,usetikzlibrary,tikzexternalize,psset,tikzset,PrependGraphicsExtensions,%
    DefineShortVerb,lstMakeShortInline,MakeSpecialShortVerb,UndefineShortVerb,verb,myverb,macro,put,line},%
% Important words 3, options in input file
    keywordstyle=[3]{\color{optcolor!85}},%
    keywords=[3]{extrenv,deltenv,skipenv,options,arara,extensions,luatex,env,norun,%
    clean,pst,tkz,eps,pdf,xetex,latex,luatex,dvips,png,srcenv,noprew,imgdir,prefix,output},%
% Important words 4, environments and tags
    keywordstyle=[4]{\color{pkgcolor}},%
    keywords=[4]{ltximg,noltximg,remove,pspicture,endpspicture,%
    description,filecontents,PSTexample,pgfpicture, endpgfpicture, tikzpicture, endtikzpicture, %
    psgraph, endpsgraph,nopreview,postscript,picture},%
% Important words 5, directory
    keywordstyle=[5]{\color{OrangeRed}},%
    keywords=[5]{images,pics},%
% Important words 6, begin, end and other
    keywordstyle=[6]{\color{gray}},% replace this color
    keywords=[6]{begin,end,-recorder,doc,active,tightpage,BEGIN,END,%
    -shell-escape,osf,--output-format,--exact-bbox,-o,dvi},%
% Important words 7, reserved for arara
    keywordstyle=[7]{\color{araracolor}},%
    keywords=[7]{lualatex,dvisvgm},%
% Important words 8, pkg and others load in examples
    keywordstyle=[8]{\color{SlateBlue}},%
    keywords=[8]{graphicx,grfext,tikz,article,document,preview,libertinus,calc,tikzmark},%
% Important words 9, files used in examples
    keywordstyle=[10]{\color{OrangeRed}},%
    keywords=[10]{test-fig-1981.tex,test-fig-1,test-fig-2,test-out,test-in.ltx,%
    test-fig-1981.tex,test-out.tex,file-in.tex,file-out,test.tex,test-fig-all.dvi,test-fig-1.svg,test-fig-all.tex},%
}[keywords,tex,comments,strings]% end languaje

% \begin{examplecode}[opts]...\end{examplecode}
\lstnewenvironment{examplecode}[1][]{%
\lstset{
    language=ltximg-doc,%
    stringstyle = {\color{red}},%
    basicstyle=\ttfamily\small,%
    numbersep=1em,%
    numberstyle=\tiny\color{gray}\noaccsupp,%
    rulecolor=\color{gray!50},%
    framesep=\fboxsep,%
    framerule=\fboxrule,%
    xleftmargin=\dimexpr\fboxsep+\fboxrule\relax,%
    xrightmargin=\dimexpr\fboxsep+\fboxrule\relax,%
% literateee
    literate=*{\{}{{\bfseries\textcolor{gray}{\{}}}{1}
              {\}}{{\bfseries\textcolor{gray}{\}}}}{1}
              {[}{{\textcolor{gray}{[}}}{1}
              {]}{{\textcolor{gray}{]}}}{1}
              {\*}{{\bfseries\textcolor{MediumOrchid}{*}}}{1}
              {:}{{\textcolor{gray}{:}}}{1}
              {,}{{\textcolor{gray}{,}}}{1}
              {=}{{\textcolor{gray}{=}}}{1}
              {/}{{\textcolor{OrangeRed}{/}}}{1}
              {scale=1}{{\textcolor{gray}{scale=1}}}{7}
              {\%\ ltximg}{{\textcolor{gray}{\%}\space\textcolor{optcolor}{ltximg}}}{8}
              {\%\ arara}{{\textcolor{gray}{\%}\space\textcolor{araracolor}{arara}}}{7}
              {\%<*remove>}{{\textcolor{gray}{\%<*}\textcolor{pkgcolor}{remove}\textcolor{gray}{>}}}{10}
              {\%</remove>}{{\textcolor{gray}{\%</}\textcolor{pkgcolor}{remove}\textcolor{gray}{>}}}{10}
              {\%<*ltximg>}{{\textcolor{gray}{\%<*}\textcolor{pkgcolor}{ltximg}\textcolor{gray}{>}}}{10}
              {\%</ltximg>}{{\textcolor{gray}{\%</}\textcolor{pkgcolor}{ltximg}\textcolor{gray}{>}}}{10}
              {\%<*noltximg>}{{\textcolor{gray}{\%<*}\textcolor{pkgcolor}{noltximg}\textcolor{gray}{>}}}{12}
              {\%</noltximg>}{{\textcolor{gray}{\%</}\textcolor{pkgcolor}{noltximg}\textcolor{gray}{>}}}{12},%
          #1,%
    }% close lstset
}%
{}% close examplecode

% \begin{examplecmd}...\end{examplecmd}
\lstnewenvironment{examplecmd}{%
\lstset{
    language=ltximg-doc,%
    basicstyle=\ttfamily\small,%
    frame=single,%
    alsoletter={\-,.,\-\-},%
    rulecolor=\color{gray!50},%
    framesep=\fboxsep,%
    framerule=\fboxrule,%
    xleftmargin=\dimexpr\fboxsep+\fboxrule\relax,%
    xrightmargin=\dimexpr\fboxsep+\fboxrule\relax,%
    keywordstyle=[20]{\color{gray}},% replace this color
    keywords=[20]{-recorder,-shell-escape,--latex,-s,-o,--arara,--norun,--luatex},%
% Reserved words (cmd line options)
    deletekeywords=[4]{ltximg},%
    classoffset=7,%
    keywordstyle=\bfseries\color{NavyBlue},%
    morekeywords={ltximg},%
% % Reserved words (cmd line options)
    classoffset=8,%
    keywordstyle={\bfseries\color{ForestGreen}},%
    morekeywords={gs,pdftoppm,pdftocairo,pdftops,cd,perl},%
% Only for command line options
    classoffset=5,%
    keywordstyle=\color{blue},%
    keywords={user,machine},%
    literate=*{[}{{\textcolor{darkgray}{[}}}{1}
              {]}{{\textcolor{darkgray}{]}}}{1}
              {@}{{\textcolor{blue}{@}}}{1}
              {\$}{{\textcolor{blue}{\$}}}{1}
              {:}{{\textcolor{blue}{:}}}{1}
              {§}{{\textcolor{red}{\$}}}{1}
              {~}{{\textcolor{blue}{\bfseries\textasciitilde}}}{1}%
    }% close lstset
}%
{}% close examplecmd

% \lstinline[style=inline]|...|
\lstdefinestyle{inline}
  {
   language=ltximg-doc,%
   basicstyle=\ttfamily\color{gray},%
   escapechar=`,%
   upquote=true,%
   deletekeywords=[8]{preview},%
   morekeywords =[4]{preview},%
   literate=*{\%}{{\bfseries\textcolor{gray}{\%}}}{1}
             {\*}{{\bfseries\textcolor{MediumOrchid}{*}}}{1}
  }

% set default style
\lstset{style=inline}

\begin{document}

\title{%
    {\fetamontotf latex environments }\\[3pt]%
    \scalebox{3.4}{\LTXimg}\\[2pt]%
    {\fetamontotf\addfontfeature{LetterSpace=12.0} to image format}\\%
    \Large
    v\fileversion{} --- \filedate\thanks{%
    This file describes a documentation for version \fileversion, last revised \filedate.}\\[25pt]%
    \author{%
    \large%
    \raisebox{-1pt}{\textcopyright}{}2013--2020 by Pablo González L%
    \thanks{E-mail: \textcolor{optcolor}{\textsf{\guillemotleft}}\email{pablgonz@yahoo.com}\textcolor{optcolor}{\textsf{\guillemotright}}}
    }%
\small
\textsc{ctan}: \url{https://www.ctan.org/pkg/ltximg}\\
\textcolor{gray}{\scriptsize\faIcon[regular]{github}}\,\,\,\url{https://github.com/pablgonz/ltximg}
\vspace*{-2cm}
}%
\date{}
\maketitle

\begin{abstract}
\ltximg{} is a \prgname{perl} \emph{script} that automates the process of
extracting and converting environments provided by \pkgname{tikz}, %
\pkgname{pstricks} and other packages from \meta[cf=optcolor]{input file} to image
formats and standalone files using \prgname{ghostscript} and %
\prgname{poppler-utils}. Generates a file with only extracted environments
and another with all extracted environments converted to \ics{includegraphics}.
\end{abstract}

\tableofcontents
\setlength{\parskip}{3pt}

\section{License}

This program is free software; you can redistribute it and/or modify
it under the terms of the \href{https://www.gnu.org/licenses/gpl-3.0.html}{GNU General Public License} as published by
the \href{https://www.fsf.org/}{Free Software Foundation}; either version 3 of the License, or
(at your option) any later version.

This program is distributed in the hope that it will be useful, but
WITHOUT ANY WARRANTY; without even the implied warranty of
MERCHANTABILITY or FITNESS FOR A PARTICULAR PURPOSE. See the \href{https://www.gnu.org/licenses/gpl-3.0.html}{GNU General Public License}
for more details.

\section{Motivation and Acknowledgments}

The original idea was to extend the functionality of the \scriptname{pst2pdf}\cite{pst2pdf}
script (only for \env{pspicture} and \env{postscript}) to work with %
\env{tikzpicture} and other environments.

The \pkgname{tikz}\cite{tikz} package allows to \emph{externalize} the environments, but, the
idea was to be able to extend this to any type of environment covering three
central points:

\begin{enumerate}[font=\small,noitemsep,leftmargin=*]

\item Generate a separate image files for environments.

\item Generate a standalone files with only the extracted environments.

\item Generate a file replacing the environments by \ics{includegraphics}.
\end{enumerate}
\thispagestyle{plain}

\newpage
\pagestyle{myheader}

From the side of \hologo{TeX} there are some packages that cover several of these
points such as the \pkgname{preview}\cite{preview}, \pkgname{xcomment}\cite{xcomment}, \pkgname{extract}\cite{extract}
and \pkgname{cachepic}\cite{cachepic} packages among others, but none covered all points.

In the network there are some solutions in \texttt{bash} that were able to
extract and convert environments, but in general they presented problems
when the document contained \emph{\enquote{verbatim style}} code or were only
available for \OSsystem{Linux}.

Analysed the situation the best thing was to create a new \emph{\enquote{script}} that
was able to cover the three points and was multi platform, the union of all
these ideas is born \ltximg.

This script would not be possible without the great work of Herbert Voß
author of \scriptname{pst2pdf}\footnote{\url{https://ctan.org/pkg/pst2pdf}}
and Heiko Oberdiek author of \scriptname{pdfcrop}\footnote{\url{https://ctan.org/pkg/pdfcrop}}.
Several parts of the code have been taken and adapted from both scripts.

Finding the correct \emph{\enquote{regular expressions}}
and writing a \emph{\enquote{good documentation}} would be the great mission (which does not
end yet).

\section{How it works}
\label{sec:howtowork}

It is important to have a general idea of how the \emph{\enquote{extraction and conversion}}
process works and the requirements that must be fulfilled so
that everything works correctly, for this we must be clear about some
concepts related to how to work with the \meta[cf=optcolor]{input file}, the \meta{verbatim content}
and the \meta{steps process}.

\section{Requirements for operation}
\label{sec:requirements}

For the complete operation of \ltximg{} you need to have a modern %
\hologo{TeX} distribution such as \hologo{TeX}Live or \hologo{MiKTeX}, have
a version equal to or greater than \liningnums{5.28} of \prgname{perl}, a
version equal to or greater than \liningnums{9.24} of \prgname{ghostscript}
and have a version equal to or greater than \liningnums{0.52} of %
\prgname{poppler-utils}.

The distribution of \hologo{TeX}Live 2020 for \OSsystem{Windows} includes %
\ltximg{} and all requirements, \hologo{MiKTeX} users must install the
appropriate software for full operation.

The script has been tested on \OSsystem{Windows} (v10) and %
\OSsystem{Linux} (fedora 32) in x64 architecture using \prgname{ghostscript} %
\liningnums{v9.52}, \prgname{poppler-utils} \liningnums{v0.84}, \prgname{perl} \liningnums{v5.30}
and the standard classes offers by \hologo{LaTeX}: \myclass{book},
\myclass{report}, \myclass{article} and \myclass{letter}. If an \meta{output file}
is generated, the \pkgname{graphicx}\cite{graphicx} and \pkgname{grfext}\cite{grfext}
packages will be needed.

\section{The input file}
\label{sec:inputfile}

The \meta[cf=optcolor]{input file} must comply with certain characteristics in order to
be processed, the content at the beginning and at the end of the \meta{input
file} is treated in a special way, before \lstinline|\documentclass| and after
\lstinline|\end{document}| can go any type of content, internally the script will
\emph{\enquote{split}} the \meta[cf=optcolor]{input file} at this points.

If the \meta[cf=optcolor]{input file} contains files using \ics{input}\marg[type=tt]{file} or
\ics{include}\marg[type=tt]{file} these will not be processed, from the side of
the \emph{script} they only represent lines within the file, if you want
them to be processed it is better to use the \scriptname{latexpand}\footnote{\url{https://www.ctan.org/pkg/latexpand}}
first and then process the file.

Like \ics{input}\marg[type=tt]{file} or \ics{include}\marg[type=tt]{file}, blank lines, vertical spaces and tab
characters are treated literally, for the \emph{script} the \meta[cf=optcolor]{input file}
is just a set of characters, as if it was a simple text file. It is
advisable to format the source code \meta[cf=optcolor]{input file} using utilities such
as \prgname{chktex}\footnote{\url{https://www.ctan.org/pkg/chktex}} and
\scriptname{latexindent}\footnote{\url{https://www.ctan.org/pkg/latexindent}}, especially if you want to
extract the source code of the environments.

Both \ics{thispagestyle}\marg[type=tt]{style} and \ics{pagestyle}\marg[type=tt]{style}
are treated in a special way by the script, if they do not appear in the
preamble then  \ics{pagestyle}\marg[type=tt]{empty} will be added and if they are present
and \marg[type=tt]{style} is different from \marg[type=tt]{empty} this
will be replaced by \marg[type=tt]{empty}.

This is necessary for the image creation process, it does not affect the
\meta{output file}, but it does affect the \emph{standalone} files.

For the script the process of dividing the \meta{input file} into four
parts and then processing them:

\begin{examplecode}[numbers=left,frame=single]
% Part One: Everything before \documentclass
\documentclass{article}
% Part two: Everything between \documentclass and \begin{document}
\begin{document}
% Part three: : Everything between \begin{document} and \end{document}
\end{document}
% Part Four: Everything after \end{document}
\end{examplecode}

\newpage

\section{Verbatim contents}
\label{sec:verbatim}

One of the greatest capabilities of this script is to \emph{\enquote{skip}}
the complications that \meta{verbatim content} produces with the extraction
of environments using tools outside the \enquote{\hologo{TeX} world}\footnote{Only \hologo{TeX} can
understand \hologo{TeX}, all other languages and programs are just lines in a file.}.
In order to \emph{\enquote{skip}} the complications, the \meta{verbatim content} is classified into
three types:

\begin{itemize}[nosep]
    \item Verbatim in line.
    \item Verbatim standard.
    \item Verbatim write.
\end{itemize}

\subsection{Verbatim in line}
\label{sec:verbatim:inline}

The small pieces of code written using a \emph{\enquote{verbatim macro}}
are considered \meta{verbatim in line}, such as \lstinline+\verb|+\meta{code}\lstinline+|+
or \lstinline+\verb*|+\meta{code}\lstinline+|+ or \lstinline+\macro+\marg[type=tt]{code} or
\lstinline+\macro+\oarg[type=tt]{opts}\marg[type=tt]{code}.

Most \emph{\enquote{verbatim macro}} provide by packages \pkgname{minted}\cite{minted}, %
\pkgname{fancyvrb}\cite{fancyvrb} and \pkgname{listings}\cite{listings} have been tested and are fully
supported. They are automatically detected the \emph{verbatim macro} (including \texttt{\small\bfseries\textcolor{MediumOrchid}{*}} argument) generates by
\ics{newmint} and \ics{newmintinline} and the following list:

\vspace*{-10pt}
\begin{multicols}{3}
\begin{itemize}[font=\sffamily\small,partopsep=5pt,parsep=5pt,nosep,leftmargin=*]
\small
\item \ics{mint}
\item \ics{spverb}
\item \ics{qverb}
\item \ics{fverb}
\item \ics{verb}
\item \ics{Verb}
\item \ics{lstinline}
\item \ics{pyginline}
\item \ics{pygment}
\item \ics{Scontents}
\item \ics{tcboxverb}
\item \ics{mintinline}
\end{itemize}
\end{multicols}

Some packages define abbreviated versions for \emph{\enquote{verbatim macro}} as %
\ics{DefineShortVerb}, \ics{lstMakeShortInline} and %
\ics{MakeSpecialShortVerb}, will be detected automatically if are declared
explicitly in \meta[cf=optcolor]{input file}.

The following consideration should be kept in mind for some packages that
use abbreviations for verbatim macros, such as \pkgname{shortvrb}\cite{doc} or %
\pkgname{doc}\cite{doc} for example in which there is no explicit \lstinline+\macro+ in the
document by means of which the abbreviated form can be detected, for
automatic detection need to find \ics{DefineShortVerb} explicitly to process
it correctly. The solution is quite simple, just add in \meta[cf=optcolor]{input file}:

\begin{examplecode}
\UndefineShortVerb{\|}
\DefineShortVerb{\|}
\end{examplecode}

depending on the package you are using. If your \emph{\enquote{verbatim macro}} is not
supported by default or can not detect, use the options described in \ref%
{sec:optline} and \ref{sec:optfile}.

\subsection{Verbatim standard}
\label{sec:verbatim:std}

These are the \emph{\enquote{classic}} environments for \emph{\enquote{writing code}} are considered %
\meta{verbatim standard}, such as \env{verbatim} and \env{lstlisting}
environments. The following list (including \texttt{\small\bfseries\textcolor{MediumOrchid}{*}} argument)
is considered as \meta{verbatim standard} environments:

\begin{multicols}{4}
\begin{itemize}[font=\sffamily\small, noitemsep,leftmargin=*]
\small
\item \env{Example}
\item \env{CenterExample}
\item \env{SideBySideExample}
\item \env{PCenterExample}
\item \env{PSideBySideExample}
\item \env{verbatim}
\item \env{Verbatim}
\item \env{BVerbatim}
\item \env{LVerbatim}
\item \env{SaveVerbatim}
\item \env{PSTcode}
\item \env{LTXexample}
\item \env{tcblisting}
\item \env{spverbatim}
\item \env{minted}
\item \env{listing}
\item \env{lstlisting}
\item \env{alltt}
\item \env{comment}
\item \env{chklisting}
\item \env{verbatimtab}
\item \env{listingcont}
\item \env{boxedverbatim}
\item \env{demo}
\item \env{sourcecode}
\item \env{xcomment}
\item \env{pygmented}
\item \env{pyglist}
\item \env{program}
\item \env{programl}
\item \env{programL}
\item \env{programs}
\item \env{programf}
\item \env{programsc}
\item \env{programt}
\end{itemize}
\end{multicols}

They are automatically detected \meta{verbatim standard} environments
(including \texttt{\small\bfseries\textcolor{MediumOrchid}{*}} argument)
generates by commands:

\begin{multicols}{2}
\begin{itemize}[font=\sffamily\small,noitemsep,leftmargin=*]
\small
\item \ics{DefineVerbatimEnvironment}
\item \ics{NewListingEnvironment}
\item \ics{DeclareTCBListing}
\item \ics{ProvideTCBListing}
\item \ics{lstnewenvironment}
\item \ics{newtabverbatim}
\item \ics{specialcomment}
\item \ics{includecomment}
\item \ics{newtcblisting}
\item \ics{NewTCBListing}
\item \ics{newverbatim}
\item \ics{NewProgram}
\item \ics{newminted}
\end{itemize}
\end{multicols}

If any of the \meta{verbatim standard} environments is not supported by
default or can not detected, you can use the options described in \ref%
{sec:optline} and \ref{sec:optfile}.

\subsection{Verbatim write}
\label{sec:verbatim:write}

Some environments have the ability to write \emph{\enquote{external files}}
or \emph{\enquote{store content}} in memory, these environments are
considered \meta{verbatim write}, such as \env{scontents}, \env{filecontents}
or \env{VerbatimOut} environments. The following list is considered (including \texttt{\small\bfseries\textcolor{MediumOrchid}{*}} argument) as %
\meta{verbatim write} environments:

\begin{multicols}{4}
\begin{itemize}[font=\sffamily\small, noitemsep,leftmargin=*]
\ttfamily\small
\item \env{scontents}
\item \env{filecontents}
\item \env{tcboutputlisting}
\item \env{tcbexternal}
\item \env{tcbwritetmp}
\item \env{extcolorbox}
\item \env{extikzpicture}
\item \env{VerbatimOut}
\item \env{verbatimwrite}
\item \env{filecontentsdef}
\item \env{filecontentshere}
\item \env{filecontentsdefmacro}
\item \env{filecontentsdefstarred}
\item \env{filecontentsgdef}
\item \env{filecontentsdefmacro}
\item \env{filecontentsgdefmacro}
\end{itemize}
\end{multicols}

They are automatically detected \meta{verbatim write}
(including \texttt{\small\bfseries\textcolor{MediumOrchid}{*}} argument) environments generates
by commands:

\begin{itemize}[font=\sffamily\small, noitemsep,leftmargin=*]
\small
\item \ics{renewtcbexternalizetcolorbox}
\item \ics{renewtcbexternalizeenvironment}
\item \ics{newtcbexternalizeenvironment}
\item \ics{newtcbexternalizetcolorbox}
\item \ics{newenvsc}
\end{itemize}

If any of the \meta{verbatim write} environments is not supported by default
or can not detected, you can use the options described in \ref{sec:optline}
and \ref{sec:optfile}.

\section{Steps process}
\label{sec:steps:process}

For creation of the image formats, extraction of source code of environments and creation of an
\meta[cf=optcolor]{output file}, \ltximg{} need a various steps. Let's assume that the %
\meta[cf=optcolor]{input file} is \sysfile{test.tex}, \meta[cf=optcolor]{output file} is %
\sysfile{test-out.tex}, the working directory are \sysdir{./}, the
directory for images are \sysdir{./images}, the temporary
directory is \sysdir{/tmp} and we want to generate images in \iext{pdf}
format and \meta{standalone files} with the source code of the environments.

We will use the following code as \sysfile{test.tex}

\begin{examplecode}[numbers=left,frame=single]
% Some commented lines at begin file
\documentclass{article}
\usepackage{tikz}
\begin{document}
Some text
\begin{tikzpicture}
  Some code
\end{tikzpicture}
Always use \verb|\begin{tikzpicture}| and \verb|\end{tikzpicture}| to open
and close environment
\begin{tikzpicture}
  Some code
\end{tikzpicture}
Some text
\begin{verbatim}
\begin{tikzpicture}
  Some code
\end{tikzpicture}
\end{verbatim}
Some text
\end{document}
Some lines that will be ignored by the script
\end{examplecode}

\subsection{Validating Options}

The first step is read and validated \oarg[type=tt,cf=optcolor,sbc=gray,ac=lightgray]{options} from the command
line and \sysfile{test.tex}, verifying that \sysfile{test.tex} contains \emph{some} environment to extract,
check the name and extension of \sysfile{test-out.tex}, check the directory \sysdir{./images} if it doesn't exist create it
and create a temporary directory \sysdir{/tmp/hG45uVklv9}.

The entire \sysfile{test.tex} file is loaded into memory and split to start the extraction process.

\newpage

\subsection{Comment and ignore}
In the second step, once the file \sysfile{test.tex} is loaded and divided in memory,
proceeds (in general terms) as follows:

\begin{quotation}
Search the words \lstinline|\begin{| and \lstinline|\end{| in verbatim standard, verbatim write, verbatim in line and
commented lines, if it finds them, converts to \lstinline|\BEGIN{| and \lstinline|\END{|, then places all code to
extract inside the \lstinline|\begin{preview}| \ldots \lstinline|\end{preview}|.
\end{quotation}

At this point \emph{\enquote{all}} the code you want to extract is inside \lstinline|\begin{preview}|\ldots\lstinline|\end{preview}|.

\subsection{Creating files and extracting}

In the third step, the script generate \meta{standalone files} \sysfile{test-fig-1.tex},
\sysfile{test-fig-2.tex} and saved in \sysdir{./images}. A temporary
file with a random number (1981 for example) with all environments is created
and proceed in two ways according to the \oarg[type=tt,cf=optcolor,sbc=gray,ac=lightgray]{options}
passed to the script:

\begin{enumerate}[leftmargin=*]
\item If script is call \emph{whitout} \cmdopt[n]{noprew} options, adds the
following lines to the beginning of the \sysfile{test.tex} (in memory):

\begin{examplecode}
\AtBeginDocument{%
\RequirePackage[active,tightpage]{preview}
\renewcommand\PreviewBbAdjust{-60pt -60pt 60pt 60pt}}%
% rest of input file
\end{examplecode}
And save in a temporary file \sysfile{test-fig-1981.tex} in \sysdir{./}.

\item If script is call \emph{whit} \cmdopt[n]{noprew} options, the \lstinline|\begin{preview}|\ldots\lstinline|\end{preview}|
lines are only used as delimiters for extracting the content \emph{without} using the package \pkgname{preview}.

Creating a temporary file \sysfile{test-fig-1981.tex} in \sysdir{./}
whit the same preamble of \sysfile{test.tex}, but the \meta{body} only contains code that you want to extract.
\end{enumerate}

If \cmdopt{norun} is passed, the temporary file \sysfile{test-fig-1981.tex}
is renamed to \sysfile{test-fig-all.tex} and moved to \sysdir{./images}.

\subsection{Generate image formats}

In the fourth step, the script run:
\begin{examplecmd}
[user@machine ~:]§ `\small\meta[type=tt,cf=ForestGreen,ac=lightgray]{compiler}` -recorder -shell-escape test-fig-1981.tex
\end{examplecmd}
generating the file \sysfile{test-fig-1981.pdf} whit all code extracted, move \sysfile{test-fig-1981.pdf}
to \sysdir{/tmp/hG45uVklv9} and rename to \sysfile{test-fig-all.pdf}, separate in individual files \sysfile{test-fig-1.pdf}
and \sysfile{test-fig-2.pdf} and copy to \sysdir{./images}. The file \sysfile{test-fig-1981.tex} is
moved to the \sysdir{./images} and rename to \sysfile{test-fig-all.tex}.

Note the options passed to \meta[type=tt,cf=ForestGreen,ac=lightgray]{compiler} does not include \prgopt*{output-directory}
(it is not supported) and always use \prgopt*{recorder} \prgopt*{shell-escape}.

\subsection{Create output file}

In the fifth step, the script creates the output file \sysfile{test-out.tex} converting all extracted code to
\ics{includegraphics} and adding the following lines at end of preamble:

\begin{examplecode}[numbers=left]
\usepackage{graphicx}
\graphicspath{{images/}}
\usepackage{grfext}
\PrependGraphicsExtensions*{.pdf}
\end{examplecode}

The script will try to detect whether the \pkgname{graphicx} package and
the \ics{graphicspath} command are in the preamble of the \meta[cf=optcolor]{output file}.
If it is not possible to find it, it will read the log file generated by the temporary file.
Once the detection is complete, the package \pkgname{grfext} and \ics*{PrependGraphicsExtensions*}
will be added at the end of the preamble, then proceed to run:

\begin{examplecmd}
[user@machine ~:]§ `\small\meta[type=tt,cf=ForestGreen,ac=lightgray]{compiler}` -recorder -shell-escape test-out.tex
\end{examplecmd}
generating the file \sysfile{test-out.pdf}.

\subsection{Clean temporary files and dirs}

In the sixth step, the script read the files \sysfile{test-fig-1981.fls}
and \sysfile{test-out.fls}, extract the information from the
temporary files and dirs generated in the process in \sysdir{./} and then
delete them together with the directory \sysdir{/tmp/hG45uVklv9}.

Finally the output file \sysfile{test-out.tex} looks like this:

\begin{examplecode}[numbers=left]
% some commented lines at begin document
\documentclass{article}
\usepackage{tikz}
\graphicspath{{images/}}
\usepackage{grfext}
\PrependGraphicsExtensions*{.pdf}
\begin{document}
Some text
\includegraphics[scale=1]{test-fig-1}
Always use \verb|\begin{tikzpicture}| and \verb|\end{tikzpicture}| to open
and close environment
\includegraphics[scale=1]{test-fig-2}
Some text
\begin{verbatim}
\begin{tikzpicture}
  Some code
\end{tikzpicture}
\end{verbatim}
Some text
\end{document}
\end{examplecode}

\section{Extract content}
\label{sec:extract}

The script provides two ways to \meta{extract} content from \meta[cf=optcolor]{input file},
using \meta[type=rm,cf=optcolor,ac=gray]{environments} and \meta[type=rm,cf=optcolor,ac=gray]{docstrip tags}.
Some environment (including \texttt{\small\bfseries\textcolor{MediumOrchid}{*}} argument)
are supported by default. If environments are nested, the outermost one will be extracted.

\subsection{Default environments}
\label{sec:extract:env}

\DescribeTE{preview}
Environment provide by \pkgname{preview}\cite{preview} package. If \env{preview} environments
found in the \meta[cf=optcolor]{input file} will be extracted and converted these. Internally
the script converts all environments to extract in \env{preview} environments.
Is better comment this package in preamble unless the option \cmdopt[n]{noprew}{} is
used.

\vspace{\baselineskip}

\DescribeTE{pspicture}
Environment provide by \pkgname{pstricks}\cite{pstricks} package. The plain \hologo{TeX}
syntax \lstinline|\pspicture ... \endpspicture| its converted to \hologo{LaTeX} syntax
\lstinline|\begin{pspicture} ... \end{pspicture}| if not within the \env{PSTexample} environment.

\vspace{\baselineskip}

\DescribeTE{psgraph}
Environment provide by \pkgname{pst-plot}\cite{pst-plot} package. The plain \hologo{TeX} \lstinline|\psgraph ... \endpsgraph|
its converted to \hologo{LaTeX} syntax \lstinline|\begin{psgraph} ... \end{psgraph}| if
not within the \env{PSTexample} environment.

\vspace{\baselineskip}

\DescribeTE{postscript}
Environment provide by \pkgname{pst-pdf}\cite{pst-pdf} and \pkgname{auto-pst-pdf}\cite{auto-pst-pdf} packages.
Since the \pkgname{pst-pdf} and \pkgname{auto-pst-pdf} packages internally use
the \pkgname{preview} package, is better comment this in preamble.

\vspace{\baselineskip}

\DescribeTE{tikzpicture}
Environment provide by \pkgname{tikz}\cite{tikz} package. The plain \hologo{TeX} \lstinline|\tikzpicture ... \tikzpicture|
its converted to \hologo{LaTeX} syntax \lstinline|\begin{tikzpicture} ... \end{tikzpicture}|
but no a short syntax \lstinline|\tikz ... ;|.
\vspace{\baselineskip}

\DescribeTE{pgfpicture}
Environment provide by \pkgname{pgf}\cite{tikz} package. Since the script uses a
\emph{\enquote{recursive regular expression}} to extract environments, no presents problems
if present \env{pgfinterruptpicture}.

\vspace{\baselineskip}

\DescribeTE{PSTexample}
Environment provide by \pkgname{pst-exa}\cite{pstexa} packages. The script automatically detects the
\lstinline|\begin{PSTexample}|  \lstinline|...\end{PSTexample}|
environments and processes them as separately compiled files. The user should have loaded the
package with the [\pkgopt{swpl}] or [\pkgopt{tcb}] option and run the script
using \cmdopt{latex}{} or \cmdopt{xetex}.

If you need to extract other environments you can use one of the options described
in \ref{sec:optline} or \ref{sec:optfile}.

\subsection{Extract whit docstrip tags}
\label{sec:extract:tag}

\DescribeTE*{ltximg}
All content included between \lstinline|%<*ltximg> ... %</ltximg>| is extracted.
The tags can \emph{not} be nested and should be at the beginning of the line and in separate lines.
Internally the script converts all this tags to \env{preview} environments.

\begin{examplecode}[frame=single]
% no space before open tag %<*
%<*ltximg>
code to extract
%</ltximg>
% no space before close tag %</
\end{examplecode}

\subsection{Prevent extraction and remove}
\label{sec:noextract}

Sometimes you do not want to \emph{\enquote{extract all}} the environments from \meta[cf=optcolor]{input file} or you want to
remove environments or arbitrary content. The script provides a convenient way to solve this situation.

\DescribeTE{nopreview}
Environment provide by \pkgname{preview} package. Internally the script
converts all \emph{\enquote{skip}} environments to \lstinline|\begin{nopreview} ... \end{nopreview}|.
Is better comment this package in preamble unless the option \cmdopt[n]{noprew}{} is used.
\vspace{\baselineskip}

\DescribeTE*{noltximg}
All content betwen \lstinline|%<*noltximg> ... %</noltximg>| are ignored and no extract.
The tags can \emph{not} be nested and should be at the beginning of the line and in separate lines.
Internally the script converts all this tags to \env{nopreview} environments.

\begin{examplecode}[frame=single]
% no space before open tag %<*
%<*noltximg>
no extract this
%</noltximg>
% no space before close tag %</
\end{examplecode}

\DescribeTE*{remove}
All content betwen \lstinline|%<*remove> ... %</remove>| are deleted in the \meta[cf=optcolor]{output file}.
The tags can \emph{not} be nested and should be at the beginning of the line and in separate lines.

\begin{examplecode}[frame=single]
% no space before open tag %<*
%<*remove>
lines removed in output file
%</remove>
% no space before close tag %</
\end{examplecode}

The content will be deleted if it is \emph{\enquote{not}} within a \meta{verbatim} or \meta{verbatim write}
environment. If you want to remove specific environments automatically you can use one of
the options described in \ref{sec:optline} or \ref{sec:optfile}.

\section{Image Formats}
\label{sec:image:format}

The \meta{image formats} generated by the \ltximg{} using \prgname{ghostscript}
and \prgname{poppler-utils} are the following command lines:

\DescribeIF{pdf}
The image format generated using \prgname{ghostscript}. The line executed by the system is:

\begin{examplecmd}
[user@machine ~:]§ gs -q -dNOSAFER -sDEVICE=pdfwrite -dPDFSETTINGS=/prepress
\end{examplecmd}

\DescribeIF{eps}
The image format generated using \prgname{pdftoeps}. The line executed by the system is:

\begin{examplecmd}
[user@machine ~:]§ pdftops -q -eps
\end{examplecmd}

\DescribeIF{png}
The image format generated using \prgname{ghostscript}. The line executed by the system is:

\begin{examplecmd}
[user@machine ~:]§ gs -q -dNOSAFER -sDEVICE=pngalpha -r 150
\end{examplecmd}

\DescribeIF{jpg}
The image format generated using \prgname{ghostscript}. The line executed by the system is:
\begin{examplecmd}
[user@machine ~:]§ gs -q -dNOSAFER -sDEVICE=jpeg -r 150 -dJPEGQ=100 \
                      -dGraphicsAlphaBits=4 -dTextAlphaBits=4
\end{examplecmd}

\DescribeIF{ppm}
The image format generated using \prgname{pdftoppm}. The line executed by the system is:

\begin{examplecmd}
[user@machine ~:]§ pdftoppm -q -r 150
\end{examplecmd}

\DescribeIF{tif}
The image format generated using \prgname{ghostscript}. The line executed by the system is:

\begin{examplecmd}
[user@machine ~:]§ gs -q -dNOSAFER -sDEVICE=tiff32nc -r 150
\end{examplecmd}

\DescribeIF{svg}
The image format generated using \prgname{pdftocairo}. The line executed by the system is:

\begin{examplecmd}
[user@machine ~:]§ pdftocairo -q -r 150
\end{examplecmd}

\DescribeIF{bmp}
The image format generated using \prgname{ghostscript}. The line executed by the system is:

\begin{examplecmd}
[user@machine ~:]§ gs -q -dNOSAFER -sDEVICE=bmp32b -r 150
\end{examplecmd}

The script auto detects the \prgname{ghostscript}, but not \prgname{poppler-utils}.
You should keep this in mind if you are using the script directly and not
the version provided in your \hologo{TeX} distribution

\section{How to use}
\subsection{Syntax}
The syntax for \ltximg{} is simple, if your use the version provided in your \hologo{TeX} distribution:

\begin{examplecmd}
[user@machine ~:]§ ltximg `\small\meta[type=tt,cf=ForestGreen,ac=lightgray]{compiler} \oarg[type=tt,cf=optcolor,sbc=gray,ac=lightgray]{options} \textcolor{gray}{\texttt{[-\/-]}} \meta[cf=optcolor]{input file}`
\end{examplecmd}
If the development version is used:
\begin{examplecmd}
[user@machine ~:]§ perl ltximg `\small\meta[type=tt,cf=ForestGreen,ac=lightgray]{compiler} \oarg[type=tt,cf=optcolor,sbc=gray,ac=lightgray]{options} \textcolor{gray}{\texttt{[-\/-]}} \meta[cf=optcolor]{input file}`
\end{examplecmd}

The extension valid for \meta[cf=optcolor]{input file} are \fext{tex} or \fext{ltx},
relative or absolute paths for files and directories is not supported. If used without \meta[type=tt,cf=ForestGreen,ac=lightgray]{compiler}
and \oarg[type=tt,cf=optcolor,sbc=gray,ac=lightgray]{options} the extracted environments are converted to \iext{pdf}
image format and saved in the \sysdir{./images} directory using \prgname*{pdflatex} and \pkgname{preview} package.

\subsection{Command line interface}
\label{sec:optline}

The script provides a \emph{command line interface} with short \textcolor{optcolor}{\bfseries\texttt{-}}
and long \textcolor{optcolor}{\bfseries\texttt{--}} option, they may be given before
the name of the \meta[cf=optcolor]{input file}, the order of specifying the options is not significant.
Options that accept a \meta{value} require either a blank space \verb*| | or \textcolor{optcolor}{\bfseries\texttt{=}}
between the option and the \meta{value}. Multiple short options can be bundling and if the last option takes a
\meta{comma separated list} you need \textcolor{optcolor}{\bfseries\texttt{-\/-}} at the end.

\DescribeCmd[h]{help}{bolean}{off}
Display a command line help and exit.

\DescribeCmd[l]{log}{bolean}{off}
Write a \fext{log} file with debug information.

\DescribeCmd[v]{version}{bolean}{off}
Display the current version (\fileversion) and exit.

\DescribeCmd[V]{verbose}{bolean}{off}
Show verbose information in terminal.

\DescribeCmd[d]{dpi}{integer}{150}
Dots per inch for images files.

\DescribeCmd[t]{tif}{bolean}{off}
Create a \fext{tif} images files using \prgname{ghostscript}.

\DescribeCmd[b]{bmp}{bolean}{off}
Create a \fext{bmp} images files using \prgname{ghostscript}.

\DescribeCmd[j]{jpg}{bolean}{off}
Create a \fext{jpg} images files using \prgname{ghostscript}.

\DescribeCmd[p]{png}{bolean}{off}
Create a \fext{png} transparent image files using \prgname{ghostscript}.

\DescribeCmd[e]{eps}{bolean}{off}
Create a \fext{eps} image files using \prgname{pdftops}.

\newpage

\DescribeCmd[s]{svg}{bolean}{off}
Create a \fext{svg} image files using \prgname{pdftocairo}.

\DescribeCmd[P]{ppm}{bolean}{off}
Create a \fext{ppm} image files using \prgname{pdftoppm}.

\DescribeCmd[g]{gray}{bolean}{off}
Create a gray scale for all images using \prgname{ghostscript}. The line behind this options is:

\begin{examplecmd}
[user@machine ~:]§ gs -q -dNOSAFER -sDEVICE=pdfwrite -dPDFSETTINGS=/prepress        \
                      -sColorConversionStrategy=Gray -dProcessColorModel=/DeviceGray
\end{examplecmd}

\DescribeCmd[f]{force}{bolean}{off}
Try to capture \lstinline|\psset|\marg[type=tt]{code} and \lstinline|\tikzset|\marg[type=tt]{code} to extract.
When using the \cmdopt{force} option the script will try to capture \lstinline|\psset|\marg[type=tt]{code} or
\lstinline|\tikzset|\marg[type=tt]{code} and leave it inside the \env{preview} environment, any line that is between \lstinline|\psset|\marg[type=tt]{code} and
\lstinline|\begin{pspicture}| or between \lstinline|\tikzset|\marg[type=tt]{code} and \lstinline|\begin{tikzpicture}| will be captured.

\DescribeCmd[n]{noprew}{bolean}{off}
Create images files without \pkgname{preview} package. The \lstinline|\begin{preview}|\ldots \lstinline|\end{preview}|
lines are only used as delimiters for extracting the content \emph{without} using the package \pkgname{preview}.
Sometimes it is better to use it together with \cmdopt{force}.

\DescribeCmd[m]{margin}{integer}{0}
Set margins in bp for \scriptname{pdfcrop}.

\DescribeCmd[o]{output}{file name}{empty}
Create \meta{file name} whit all extracted environments converted to \ics{includegraphics}.
Only \meta{file name} must be passed \emph{without} relative or absolute paths.

\DescribeCmd{prefix}{string}{fig}
Set \meta{prefix} append to each generated files.

\DescribeCmd{verbcmd}{macro name}{myverb}
Set custom verbatim command \lstinline+\myverb+. Just pass the \meta{name}
of the macro \emph{without} \enquote{\textcolor{optcolor}{\texttt{\bfseries\textbackslash}}}.

\DescribeCmd{imgdir}{string}{images}
Set the name of directory for save generated files. Only the name of directory
must be passed without \emph{without} or absolute paths.

\DescribeCmd{zip}{bolean}{off}
Compress the files generated by the script in \sysdir{./images} in \fext{zip} format.
Does not include \meta[cf=optcolor]{output file}.

\DescribeCmd{tar}{bolean}{off}
Compress the files generated by the script in \sysdir{./images} in \fext{tar.gz} format.
Does not include \meta[cf=optcolor]{output file}.

\DescribeCmd{srcenv}{bolean}{off}
Create separate files whit \emph{\enquote{only code}} for all extracted
environments. This option is designed to generate \meta{standalone files}
with \emph{\enquote{only code}} of the environments, is mutually exclusive whit \cmdopt{subenv}.

\DescribeCmd{subenv}{bolean}{off}
Create sub files whit \emph{\enquote{preamble and code}} for all extracted
environments. This option is designed to generate \meta{standalone compilable files}
for each extracted environment, is mutually exclusive whit \cmdopt{srcenv}.

\DescribeCmd{norun}{bolean}{off}
Execute the script, but do not create image files. This option is designed to
be used in conjunction with \cmdopt{srcenv} or \cmdopt{subenv} and to
debug the \meta[cf=optcolor]{output file}.

\DescribeCmd{nopdf}{bolean}{off}
Don't create a \fext{pdf} image files.

\DescribeCmd{nocrop}{bolean}{off}
Don't run \scriptname{pdfcrop} in image files.

\DescribeCmd{arara}{bolean}{off}
Use \prgname*{arara} for compiler files.

\DescribeCmd{xetex}{bolean}{off}
Using \prgname*{xelatex} compiler \meta[cf=optcolor]{input file} and \meta[cf=optcolor]{output file}.

\DescribeCmd{latex}{bolean}{off}
Using \prgname*{latex}\textcolor{optcolor}{\texttt{\bfseries\guillemotright}}\prgname*{dvips}\textcolor{optcolor}{\texttt{\bfseries\guillemotright}}\scriptname{ps2pdf}
compiler in \meta[cf=optcolor]{input file} and \prgname*{pdflatex} for \meta[cf=optcolor]{output file}.

\DescribeCmd{dvips}{bolean}{off}
Using \prgname*{latex}\textcolor{optcolor}{\texttt{\bfseries\guillemotright}}\prgname*{dvips}\textcolor{optcolor}{\texttt{\bfseries\guillemotright}}\scriptname{ps2pdf}
for compiler \meta[cf=optcolor]{input file} and \meta[cf=optcolor]{output file}.

\DescribeCmd{dvipdf}{bolean}{off}
Using \prgname*{latex}\textcolor{optcolor}{\texttt{\bfseries\guillemotright}}\prgname*{dvipdfmx}
for compiler \meta[cf=optcolor]{input file} and \meta[cf=optcolor]{output file}.

\DescribeCmd{luatex}{bolean}{off}
Using \prgname*{lualatex} for compiler \meta[cf=optcolor]{input file} and \meta[cf=optcolor]{output file}.

\DescribeCmd{clean}{\textnormal{\ttfamily doc\textbar pst\textbar tkz\textbar all\textbar off}}{doc}
Removes specific content in \meta[cf=optcolor]{output file}. Valid values for this option are:

\begin{description}[font=\ttfamily, nosep, noitemsep, leftmargin=0pt]
\item[\textcolor{gray}{doc}] All content after \lstinline+\end{document}+ is removed.
\item[\textcolor{gray}{pst}] All \lstinline|\psset|\marg[type=tt]{code} and \pkgname{pstricks} package is removed in \meta{preamble} and \meta{body}.
\item[\textcolor{gray}{tkz}] All \lstinline|\tikzset|\marg[type=tt]{code} is removed in \meta{body}.
\item[\textcolor{gray}{all}] Activates \textcolor{gray}{\texttt{doc}}, \textcolor{gray}{\texttt{pst}} and \textcolor{gray}{\texttt{tkz}}.
\item[\textcolor{gray}{off}] Deactivate all.
\end{description}

\DescribeCmd{extrenv}{comma separated list}{empty}
Add environments to extract, if it's the last option passed need \textcolor{optcolor}{\textbf{\texttt{-\/-}}}
at the end.\par

\DescribeCmd{skipenv}{comma separated list}{empty}
Add environments that should \emph{\enquote{not be extracted}} and that the script supports
by default, if it's the last option passed need \textcolor{optcolor}{\textbf{\texttt{-\/-}}}
at the end.\par

\DescribeCmd{verbenv}{comma separated list}{empty}
Add \meta{verbatim standard} environment support, if it's the last option
passed need \textcolor{optcolor}{\textbf{\texttt{-\/-}}} at the end.\par

\DescribeCmd{writenv}{comma separated list}{empty}
Add \meta{verbatim write} environment support, if it's the last option
passed need \textcolor{optcolor}{\textbf{\texttt{-\/-}}} at the end.\par

\DescribeCmd{deltenv}{comma separated list}{empty}
Add environments to deleted in \meta[cf=optcolor]{output file}. The environments are
delete only in \meta{body}, if it's the last option
passed need \textcolor{optcolor}{\textbf{\texttt{-\/-}}} at the end.\par

\subsection{Options from input file}
\label{sec:optfile}

Many of the ideas in this section are inspired by the \arara\footnote{\url{https://ctan.org/pkg/arara}} tool.
A very useful way to pass options to the script is to place them in commented
lines at the beginning of the file, very much in the \emph{\enquote{style of \arara}}.

\DescribeOptFile*{\meta[ac=lightgray,cf=optcolor]{argument}}{option one, option two, option three, \ldots}
\DescribeOptFile*{\meta[ac=lightgray,cf=optcolor]{argument}}{option one, option two, option three, \ldots}[\textcolor{gray}{!}]

The vast majority of the \meta[ac=lightgray,cf=optcolor]{options} can be passed into the \meta[cf=optcolor]{input file}. These
should be put at the beginning of the file in commented lines and everything must
be on the same line, the exclamation mark \textcolor{optcolor}{\texttt{!}} deactivates the \meta{options}.
When passing options from the \meta[cf=optcolor]{input file} you should be aware that
they must \emph{\enquote{not}} contain \textcolor{optcolor}{\texttt{-}} or \textcolor{optcolor}{\texttt{-\/-}},
the \textcolor{optcolor}{\texttt{=}} sign between an option and its value is mandatory,
short names are disabled and options found in the \meta[cf=optcolor]{input file} overwrite
those passed on the command line.

Valid values for \meta[ac=lightgray,cf=optcolor]{argument} are the following:

\DescribeOptFile{options}{option one = value, option two = value, option three = value, \ldots}
This line is to indicate to the script which options need to process.

\DescribeOptFile{extrenv}{environment one, environment two, environment three, \ldots}
This line is to indicate to the script which environments, not supported by
default, are extracted.

\DescribeOptFile{skipenv}{environment one, environment two, environment three, \ldots}
This line is to indicate to the script which environments, of the ones
supported by default, should not be extracted.

\DescribeOptFile{verbenv}{environment one, environment two, environment three, \ldots}
This line is to indicate to the script which environments, its considerate a \meta{verbatim standard}.

\DescribeOptFile{writenv}{environment one, environment two, environment three, \ldots}
This line is to indicate to the script which environments its consider \meta{verbatim write}.

\DescribeOptFile{deltenv}{environment one, environment two, environment three, \ldots}
This line is to indicate to the script which environments are deleted.

If you are going to create an \meta[cf=optcolor]{output file} and you do not want these lines to remain, it is better to
place them inside the \lstinline|%<*remove> ... %</remove>|. Like this:

\begin{examplecode}[numbers=left]
%<*remove>
% ltximg: options: {png,srcenv,xetex}
% ltximg: extrenv: {description}
%</remove>
\end{examplecode}

\section{The way of \texttt{arara} and \texttt{dvisvgm}}
\label{sec:arara}

By design, the script only runs \emph{\enquote{one compilation}} on top
of the \meta[cf=optcolor]{input file}, but, sometimes you need to process the \meta[cf=optcolor]{input file}
more than once or needs to be processed with something other than \hologo{LaTeX},
\hologo{XeLaTeX}, \hologo{pdfLaTeX} or \hologo{LuaLaTeX} engine.

This is where \arara\cite{arara} comes in, this \emph{\enquote{great little tool}},
is able to have complete control over the compilation of the \meta[cf=optcolor]{input file}, we just
have to keep a few considerations in mind:

\begin{enumerate}[font=\small, nosep, noitemsep,leftmargin=*]
\item Read the documentation (this always comes first).
\item Add \lstinline|{ options: [-recorder] }| to \enquote{rule} for clean temporary files.
\item Avoiding the use of \lstinline|clean: { extensions: [...] }|.
\item Don't set \lstinline|-jobname| and \lstinline|-output-directory|.
\item Remember that the \meta[cf=optcolor]{input file} and \meta[cf=optcolor]{output file} will be compiled using the same \enquote{rule}.
\end{enumerate}

With all these considerations in mind it is possible to extract and convert
from any file, for example, by adding these lines:

\begin{examplecode}[numbers=left]
% arara: lualatex: { options: [-recorder] }
% arara: lualatex: { options: [-recorder] }
%<*remove>
% ltximg: options: {arara, output = file-out, noprew, prefix = tkz}
%</remove>
\end{examplecode}
and run:
\begin{examplecmd}
[user@machine~:]§ ltximg test.tex
\end{examplecmd}

Create a \sysdir{./images} directory (if it does not exist) whit all \env{tikzpicture}
environments converted to image format (\iext{pdf}) in individual files,
an \meta[cf=optcolor]{output file} \sysfile{file-out.tex} whit all \env{tikzpicture} environments
converted to \ics{includegraphics}, a single file \sysfile{test-tkz-all.tex}
with only the extracted environments using \prgname*{lualatex} two times for \meta[cf=optcolor]{input file} and \meta[cf=optcolor]{output file}.

In this way we can \meta{compile} and \meta{convert} any document as long as the conditions
of the \meta[cf=optcolor]{input file} are met and the correct \enquote{rule} are used,

Also, by design, the image format \iext{svg} is created using \prgname{pdftocairo} over the
generated \iext{pdf} file, but, if you want to have a good \iext{svg} file it is best to
use \prgname{dvisvgm}\footnote{\url{https://ctan.org/pkg/dvisvgm}} which
is included in every modern \hologo{TeX} distribution. The best
results of \prgname{dvisvgm}\cite{dvisvgm} are obtained when processing the file in \fext{dvi} or \fext{xdv}
format, there are three possible ways to do this:

\begin{enumerate}[font=\small, nosep, noitemsep,leftmargin=*]
\item Pass the necessary options to \prgname{arara} and let him do the job.
\item Execute the script using \cmdopt{subenv}{} and \cmdopt{norun}{}, move to \sysdir{./images}
and generate \fext{dvi}(\fext{xdv}) files, then:
\begin{examplecmd}
[user@machine~:]§ for i in *.dvi; do `\small\prgname{dvisvgm} \oarg[type=tt,cf=gray,sbc=optcolor,ac=gray]{options}` §i;done
\end{examplecmd}

\item Execute the script using \cmdopt{norun}{}, move to \sysdir{./images} and generate \sysfile{test-fig-all}\fext{dvi}(\fext{xdv}), then:
\begin{examplecmd}
[user@machine~:]§ `\small\prgname{dvisvgm} \oarg[type=tt,cf=gray,sbc=optcolor,ac=gray]{options}` test-fig-all.dvi
\end{examplecmd}
\end{enumerate}

With this we can generate \iext{svg} files that preserve our typographic fonts.

\section{Examples}
\label{sec:examples}

\subsection{Passing command line options}

\begin{examplecmd}
[user@machine ~:]§ ltximg --latex -s -o test-out  test-in.ltx
\end{examplecmd}
Create a \sysdir{./images} directory (if it does not exist) whit all extracted environments converted to image
formats (\iext{pdf}, \iext{svg}) in individual files, an \meta[cf=optcolor]{output file} \sysfile{test-out.ltx}
whit all extracted environments converted to \ics{includegraphics} and a single file
\sysfile{test-in-fig-all.ltx} with only the extracted environments
using \prgname*{latex}\textcolor{optcolor}{\texttt{\bfseries\guillemotright}}\prgname*{dvips}\textcolor{optcolor}{\texttt{\bfseries\guillemotright}}\scriptname{ps2pdf}
and \pkgname{preview} package for \meta[cf=optcolor]{input file} and \prgname*{pdflatex} for \meta[cf=optcolor]{output file}.

\subsection{Passing options from input file}

Adding the following lines to the beginning of the file \sysfile{file-in.tex}:

\begin{examplecode}[numbers=left]
% ltximg: options: {luatex, output = file-out, imgdir = pics, prefix = env}
% ltximg: skipenv: {tikzpicture}
% ltximg: deltenv: {filecontents}
\end{examplecode}
and run:
\begin{examplecmd}
[user@machine~:]§ ltximg file-in.tex
\end{examplecmd}
Create a \sysdir{./pics} directory (if it does not exist) whit all extracted environments,
except \env{tikzpicture}, converted to image formats (\iext{pdf}) in individual files,
an \meta[cf=optcolor]{output file} \sysfile{file-out.tex} whit all extracted environments
converted to \ics{includegraphics} and environment \env{filecontents} removed,
a single file \sysfile{test-in-env-all.ltx} with only the extracted environments
using \prgname*{lualatex} and \pkgname{preview} package for \meta[cf=optcolor]{input file} and \meta[cf=optcolor]{output file}.

\subsection{Using \texttt{arara} and \texttt{dvisvgm}}

The process and operations required to generate the various types of \meta{image
formats} or \meta{standalone files} have been described throughout the documentation,
but, as discussed in section \ref{sec:arara}, sometimes the requirements are a \emph{little
different}. Let's take as an example\footnote{Adapted from \href{https://tex.stackexchange.com/a/209059}{How to get tikzmark to work}}
the next code which requires two compilations and use \arara{} to generate an image in \iext{pdf} format. We'll save our example as \sysfile{test.tex}.

\begin{examplecode}[numbers=left]
% arara: lualatex: { options: [-recorder] }
% arara: lualatex: { options: [-recorder] }
\documentclass{article}
\usepackage[osf]{libertinus}
\usepackage{tikz}
\usetikzlibrary{calc,tikzmark}
\begin{document}

By taking logarithms of both sides:

%<*ltximg>
\[
  t = \frac{30\cdot\ln(3/22)}{\ln(15/22)}
  \tikzmark{calculator}\approx\tikzmark{otherside}
    156
\]
\begin{tikzpicture}[overlay,remember picture]
  \coordinate (target) at ($(pic cs:calculator)!1/2!(pic cs:otherside) - (0,.5ex)$);
  \draw[arrows=->] (target) ++(0,-2ex) node [anchor=north] {use calculator} -- (target);
\end{tikzpicture}
%</ltximg>
\end{document}
\end{examplecode}

Now we just run:
\begin{examplecmd}
[user@machine~:]§ ltximg --arara test.tex
\end{examplecmd}

And we already have our image \sysfile{test-fig-1.pdf} on \sysdir{./images}.

\newpage

Let's modify the example\footnote{Adapted from \href{https://tex.stackexchange.com/a/114847}{Draw an aircraft with Tikz}}
to generate an image in \iext{svg} format
using \prgname{dvisvgm} and \arara.

\begin{examplecode}[numbers=left]
% arara: lualatex: { options: [--output-format=dvi] }
% arara: dvisvgm: { options: [--exact-bbox, -o test-fig-1.svg] }
% ltximg: extrenv: {picture}
% ltximg: options: {arara,norun,noprew}
\documentclass{article}
\begin{document}

The best airplane ever drawn by David Carlise. No packages used, just the classic and perhaps
forgotten \verb|\begin{picture} ... \end{picture}|.

\begin{picture}(200,100)
  \put(30,40){\line(1,0){150}}
  \put(30,40){\line(0,1){60}}
  \put(30,100){\line(1,0){20}}
  \put(50,100){\line(1,-4){10}}
  \put(60,60){\line(1,0){100}}
  \put(160,60){\line(1,-1){20}}
  \put(100,50){\line(0,-1){80}}
  \put(130,50){\line(0,-1){80}}
  \put(100,-30){\line(1,0){30}}
  \put(100,61){\line(0,1){49}}
  \put(130,61){\line(0,1){49}}
  \put(100,110){\line(1,0){30}}
\end{picture}
\end{document}
\end{examplecode}

We now run:

\begin{examplecmd}
[user@machine~:]§ ltximg test.tex
[user@machine~:]§ cd images/
[user@machine~:]§ `\small\prgname{arara}` test-fig-all.tex
\end{examplecmd}

And we already have our image \sysfile{test-fig-1.svg}.

Here are some of the ways you can use \ltximg{} in conjunction with \arara.
If you are a user of \prgname{latexmk}\footnote{\url{https://www.ctan.org/pkg/latexmk}}, another
great utility that automates the compilation process,  the procedure is analogous, you
just have to search (or write) the necessary \emph{\enquote{rule}}.

\newpage

\section{Change history}
\label{sec:change:history}

The most recent publicly released version of \ltximg{} is available at
\textsc{ctan}: \url{https://www.ctan.org/pkg/ltximg}. Historical and
developmental versions are available at \textcolor{gray}{\scriptsize\faIcon[regular]{github}}
\url{https://github.com/pablgonz/ltximg}.

While general feedback via email is welcomed, specific bugs or feature
requests should be reported through the issue
tracker: \url{https://github.com/pablgonz/ltximg/issues}.

This is a short list of some of the notable changes in the history of the \ltximg{} along with the
versions, both development (devp) and public (ctan).

\setlist[itemize,1]{label=\textendash,wide=0.5em,nosep,noitemsep,leftmargin=10pt}
% length for change history
\newlength\descrwidth
\settowidth{\descrwidth}{\textsf{v1.4.5, (ctan), 2013-01-23}}
\begin{description}[font=\small\sffamily,wide=0pt,style=multiline,leftmargin=\descrwidth,nosep,noitemsep]
\item [v\fileversion{} (ctan), \filedate]
    \begin{itemize}
        \item It is now possible to extract any environment.
        \item Fix \texttt{verbose} option.
        \item Add \texttt{log} option.
        \item All calls to the system are captured.
        \item Re-write source code acording to \texttt{perl} v5.3x.
        \item Review of documentation.
    \end{itemize}
\item [v1.7 (ctan), 2019-08-24]
    \begin{itemize}
        \item Add \texttt{scontents} environment support.
        \item Add \texttt{filecontentsdefmacro} environment support.
        \item Fix regex in source code.
        \item Update documentation.
    \end{itemize}
\item [v1.6 (ctan), 2019-07-13]
    \begin{itemize}
        \item Add \texttt{zip} and \texttt{tar} options.
        \item Add new \texttt{Verb} from \texttt{fvextra}.
        \item Fix and update source code and documentation.
    \end{itemize}
\item [v1.5 (ctan), 2018-04-12]
    \begin{itemize}
        \item Use \texttt{GitHub} to control version.
        \item Rewrite and optimize most part of source code and options.
        \item Change \texttt{pdf2svg} for \texttt{pdftocairo}.
        \item Complete support for \texttt{pst-exa} package.
        \item Escape characters in regex according to \texttt{perl} v5.2x.
    \end{itemize}
\item [v1.4 (devp), 2016-11-29]
    \begin{itemize}
        \item Remove and rewrite code for regex and system call.
        \item Add \texttt{arara} compiler, clean and comment code.
        \item Add \texttt{dvips} and \texttt{dvipdfm(x)} for creation images.
        \item Add \texttt{bmp}, \texttt{tiff} image format.
    \end{itemize}
\item [v1.3 (devp), 2016-08-14]
    \begin{itemize}
        \item Rewrite some part of code (norun, nocrop, clean).
        \item Suport \texttt{minted} and \texttt{tcolorbox} package.
        \item Escape some characters in regex according to \texttt{perl} v5.2x.
        \item All options read from command line and input file.
        \item Use \texttt{/tmp} dir for work process.
    \end{itemize}
\item [v1.2 (ctan), 2015-04-22]
    \begin{itemize}
        \item Remove unused modules.
        \item Add more image format.
        \item Fix regex.
    \end{itemize}
\item [v1.1 (ctan), 2015-04-21]
    \begin{itemize}
        \item Change \texttt{mogrify} to \texttt{gs} for image formats.
        \item Create output file.
        \item Rewrite source code and fix regex.
        \item Change format date to iso format.
    \end{itemize}
\item [v1.0 (ctan), 2013-12-01]
    \begin{itemize}
        \item First public release.
    \end{itemize}
\end{description}

\newpage

\let\stdsection\section
\def\section*#1{\stdsection{#1}}
\begin{thebibliography}{9}

\bibitem{preview} \textsc{Kastrup, David}. \enquote{The \textsf{preview}
package for \hologo{LaTeX}}. Available from
\textsc{ctan}, \url{https://www.ctan.org/pkg/preview}, 2017.

\bibitem{tikz} \textsc{Tantau, Till}. \enquote{The \textsf{TikZ} and \textsf{PGF}
Packages}. Available from
\textsc{ctan}, \url{https://www.ctan.org/pkg/pgf}, 2020.

\bibitem{pstricks} \textsc{Van Zandt, Timothy}. \enquote{\textsf{PSTricks} -
\textsf{PostScript} macros for generic \hologo{TeX}}. Available from
\textsc{ctan}, \url{https://www.ctan.org/pkg/pstricks-base}, 2007.

\bibitem{pst-plot} \textsc{Van Zandt, Timothy}. \enquote{\textsf{pst-plot} –
Plot data using \textsf{PSTricks}}. Available from
\textsc{ctan}, \url{https://www.ctan.org/pkg/pst-plot}, 2019.

\bibitem{pst-pdf} \textsc{Niepraschk, Rolf}. \enquote{The \textsf{pst-pdf}
Packages}. Available from
\textsc{ctan}, \url{https://www.ctan.org/pkg/pst-pdf}, 2019.

\bibitem{auto-pst-pdf} \textsc{Robertson, Will}. \enquote{The \textsf{auto-pst-pdf}
Packages}. Available from
\textsc{ctan}, \url{https://www.ctan.org/pkg/auto-pst-pdf}, 2009.

\bibitem{pstexa} \textsc{Voß, Herbert}. \enquote{\textsf{pst-exa} - Typeset
\textsf{PSTricks} examples, with \hologo{pdfTeX}}. Available from
\textsc{ctan}, \url{https://www.ctan.org/pkg/pst-exa}, 2017.

\bibitem{pst2pdf} \textsc{Voß, Herbert}. \enquote{\textsf{pst2pdf} - A script
to compile \textsf{PSTricks} documents via \hologo{pdfTeX}}. Available from
\textsc{ctan}, \url{https://www.ctan.org/pkg/pst2pdf}, 2017.

\bibitem{graphicx} \textsc{The \hologo{LaTeX3} Project}. \enquote{\textsf{graphics} – Enhanced
support for graphics}. Available from
\textsc{ctan}, \url{https://www.ctan.org/pkg/graphicx}, 2017.

\bibitem{grfext} \textsc{Oberdiek, Heiko}. \enquote{The \textsf{grfext}
package}. Available from
\textsc{ctan}, \url{https://www.ctan.org/pkg/grfext}, 2019.

\bibitem{xcomment} \textsc{Van Zandt, Timothy}. \enquote{The \textsf{xcomment}
package}. Available from
\textsc{ctan}, \url{https://www.ctan.org/pkg/xcomment}, 2010.

\bibitem{extract} \textsc{Adriaens, Hendri}. \enquote{The \textsf{extract}
package}. Available from
\textsc{ctan}, \url{https://www.ctan.org/pkg/extract}, 2019.

\bibitem{cachepic} \textsc{Trzeciak, Tomasz M}. \enquote{The \textsf{cachepic}
package}. Available from
\textsc{ctan}, \url{https://www.ctan.org/pkg/cachepic}, 2009.

\bibitem{doc} \textsc{Mittelbach, Frank}. \enquote{The \textsf{doc} and \textsf{shortvrb}
Packages}. Available from
\textsc{ctan}, \url{https://www.ctan.org/pkg/doc}, 2020.

\bibitem{fancyvrb} \textsc{Van Zandt, Timothy}. \enquote{The \textsf{fancyvrb}
package - Fancy Verbatims in \hologo{LaTeX}}. Available from
\textsc{ctan}, \url{https://www.ctan.org/pkg/fancyvrb}, 2020.

\bibitem{listings} \textsc{Hoffmann, Jobst}. \enquote{The \textsf{listings}
package}. Available from
\textsc{ctan}, \url{https://www.ctan.org/pkg/listings}, 2020.

\bibitem{minted} \textsc{Poore, Geoffrey M}. \enquote{The \textsf{minted}
package - Highlighted source code in \hologo{LaTeX}}. Available from
\textsc{ctan}, \url{https://www.ctan.org/pkg/minted}, 2017.

\bibitem{arara} \textsc{The Island of \hologo{TeX}}. \enquote{\textsf{arara} - The
cool \hologo{TeX} automation tool}. Available from
\textsc{ctan}, \url{https://www.ctan.org/pkg/arara}, 2020.

\bibitem{dvisvgm} \textsc{Gieseking, Martin}. \enquote{\textsf{dvisvgm} - A
fast \textsf{DVI} to \textsf{SVG} converter}. Available from
\textsc{ctan}, \url{https://ctan.org/pkg/dvisvgm}, 2020.

\end{thebibliography}
\let\section\stdsection

\addtocontents{toc}{\protect\setcounter{tocdepth}{2}}
\cleardoublepage
\phantomsection
\indexprologue{
The italic numbers denote the pages where the corresponding entry is
described.}

\printindex
\end{document}
%</documentation>
